\documentclass{article}
\usepackage[utf8]{inputenc}
\usepackage{amsmath}
\usepackage{graphicx}
\usepackage{xcolor}
\usepackage{tikz}
\usepackage{geometry}
\usepackage{utp-doc}

%\title{AVS.01.01.01: Análisis Metrológico de un Vehículo Eléctrico Compacto}
%\author{Profesor [Su Nombre]}
%\date{\today}

\begin{document}

%
% --- Título de la Lectura ---
\practicatitle{AVS.01.01.01: Análisis Metrológico de un Vehículo Eléctrico Compacto}

% --- Metadatos de la Actividad ---
\textbf{Asignatura:} Metrología \\
\textbf{Unidad I:} - Introducción a la Metrología 

\vspace{5mm}
\hrule
\vspace{5mm}

%\maketitle

\section*{Introducción}

En el contexto del \hl{desarrollo de vehículos eléctricos}, la \hl{metrología} juega un papel fundamental en el aseguramiento de la \hl{calidad, precisión y seguridad} de todos los componentes y sistemas. La correcta aplicación de \hl{sistemas de unidades, conversiones de magnitudes} y el uso de \hl{múltiplos y submúltiplos} es esencial para garantizar que las especificaciones técnicas cumplan con los \hl{estándares internacionales} de la industria automotriz.

El presente caso de estudio permite al estudiante \hl{aplicar los conceptos fundamentales de metrología} en un escenario real de la industria automotriz, donde deberá \hl{resolver problemas prácticos} relacionados con conversiones de unidades, análisis de especificaciones técnicas y toma de decisiones basadas en \hl{mediciones precisas} para el desarrollo de un vehículo eléctrico compacto.

\section*{Objetivo General}

Aplicar los \hl{conceptos fundamentales de metrología} incluyendo \hl{sistemas de unidades, múltiplos, submúltiplos y conversiones} en el análisis de especificaciones técnicas de un vehículo eléctrico compacto, para \hl{garantizar la precisión y calidad} en los procesos de diseño y fabricación, cumpliendo con los \hl{estándares de la industria automotriz}.

\section*{Actividad Previa al Desarrollo del Caso}

Antes de desarrollar el caso de estudio, revisa el \hl{material teórico} proporcionado y contesta el siguiente \hl{cuestionario}. Esto te ayudará a contextualizar los conceptos y a aprovechar mejor el análisis del caso.

\begin{enumerate}
    \item \textbf{¿Qué es la metrología y por qué es fundamental en la industria automotriz?}
    \item \textbf{Menciona las 7 magnitudes fundamentales del Sistema Internacional de Unidades (SI) y su aplicación en vehículos eléctricos.}
    \item \textbf{Explique la diferencia entre precisión y exactitud en las mediciones, dando un ejemplo automotriz.}
    \item \textbf{¿Por qué es importante el uso correcto de múltiplos y submúltiplos en las especificaciones técnicas automotrices?}
    \item \textbf{Mencione tres errores comunes en conversiones de unidades que podrían afectar la seguridad de un vehículo.}
\end{enumerate}

\section*{Descripción del Caso de Estudio}

\subsection*{Escenario}

Tu equipo de ingeniería ha sido contratado por \textbf{``EcoMotion México''}, una startup mexicana especializada en el desarrollo de vehículos eléctricos urbanos. La empresa está desarrollando su primer modelo: el \textbf{``UrbanE''}, un vehículo eléctrico compacto diseñado para la movilidad urbana sostenible en ciudades mexicanas.

\subsection*{Situación Problemática}

Durante la fase de diseño y especificación técnica del UrbanE, el equipo de ingeniería ha recibido información técnica de múltiples proveedores internacionales que utilizan \hl{diferentes sistemas de unidades}. Además, se han detectado \hl{inconsistencias en las especificaciones} que podrían comprometer la seguridad, eficiencia y cumplimiento normativo del vehículo.

\subsection*{Información Técnica Disponible}

\subsubsection*{Especificaciones del Motor Eléctrico (Proveedor Alemán)}
\begin{tabular}{|l|l|}
\hline
\textbf{Característica} & \textbf{Valor} \\
\hline
Potencia máxima & $75 \, \text{kW}$ \\
Torque máximo & $200 \, \text{N} \cdot \text{m}$ \\
Velocidad máxima & $8,500 \, \text{rpm}$ \\
Eficiencia & $94\%$ \\
Peso & $45 \, \text{kg}$ \\
\hline
\end{tabular}

\subsubsection*{Especificaciones de la Batería (Proveedor Chino)}
\begin{tabular}{|l|l|}
\hline
\textbf{Característica} & \textbf{Valor} \\
\hline
Capacidad & $40 \, \text{kWh}$ \\
Voltaje nominal & $400 \, \text{V}$ \\
Corriente máxima & $150 \, \text{A}$ \\
Peso & $280 \, \text{kg}$ \\
Dimensiones & $1,200 \, \text{mm} \times 800 \, \text{mm} \times 150 \, \text{mm}$ \\
\hline
\end{tabular}

\subsubsection*{Especificaciones del Chasis (Proveedor Estadounidense)}
\begin{tabular}{|l|l|}
\hline
\textbf{Característica} & \textbf{Valor} \\
\hline
Longitud total & $12.5 \, \text{ft}$ \\
Ancho & $5.8 \, \text{ft}$ \\
Altura & $5.2 \, \text{ft}$ \\
Distancia entre ejes & $8.2 \, \text{ft}$ \\
Peso estructura & $1,200 \, \text{lbs}$ \\
Capacidad de carga & $800 \, \text{lbs}$ \\
\hline
\end{tabular}

\subsubsection*{Especificaciones de Neumáticos (Proveedor Europeo)}
\begin{tabular}{|l|l|}
\hline
\textbf{Característica} & \textbf{Valor} \\
\hline
Medida & 185/65 R15 \\
Presión recomendada & $2.2 \, \text{bar}$ \\
Carga máxima por neumático & $475 \, \text{kg}$ \\
Velocidad máxima & $180 \, \text{km/h}$ \\
\hline
\end{tabular}

\subsubsection*{Requerimientos del Cliente}
\begin{tabular}{|l|l|}
\hline
\textbf{Requerimiento} & \textbf{Valor} \\
\hline
Velocidad máxima del vehículo & $120 \, \text{km/h}$ \\
Autonomía mínima & $300 \, \text{km}$ \\
Capacidad de pasajeros & 4 personas ($70 \, \text{kg}$ promedio c/u) \\
Tiempo de carga & máximo 6 horas (carga estándar) \\
Cumplimiento normativo & NOM-194-SCFI \\
\hline
\end{tabular}

\section*{Desarrollo del Caso de Estudio}

\subsection*{Fase 1: Análisis y Conversión de Unidades (40 puntos)}

\subsubsection*{1.1 Conversión de Especificaciones del Chasis}
Convierte todas las medidas del chasis del sistema imperial al Sistema Internacional (SI):
\begin{enumerate}
    \item \textbf{Dimensiones principales} (longitud, ancho, altura, distancia entre ejes) en metros ($m$).
    \item \textbf{Pesos} (estructura y capacidad de carga) en kilogramos ($kg$).
    \item \textbf{Crea una tabla comparativa} con las medidas originales y convertidas.
\end{enumerate}

\subsubsection*{1.2 Análisis de Compatibilidad de Neumáticos}
Analiza la compatibilidad de los neumáticos con las especificaciones del vehículo:
\begin{enumerate}
    \item Convierte la \textbf{presión recomendada} de bar a Pascales ($\text{Pa}$) y libras por pulgada cuadrada ($\text{psi}$).
    \item Verifica si la \textbf{carga máxima por neumático} es suficiente para el peso total del vehículo.
    \item Compara la \textbf{velocidad máxima} del neumático con los requerimientos del cliente.
\end{enumerate}

\subsection*{Fase 2: Cálculos de Rendimiento y Eficiencia (40 puntos)}

\subsubsection*{2.1 Análisis Energético}
Realiza los siguientes cálculos energéticos:
\begin{enumerate}
    \item \textbf{Densidad energética} de la batería ($\text{Wh/kg}$).
    \item \textbf{Consumo energético} estimado por kilómetro ($\text{kWh/100km}$).
    \item \textbf{Autonomía teórica} basada en la capacidad de la batería.
    \item \textbf{Tiempo de carga} a diferentes potencias: $3.3 \, \text{kW}$, $7.4 \, \text{kW}$, $22 \, \text{kW}$.
\end{enumerate}

\subsubsection*{2.2 Análisis de Potencia y Torque}
Calcula y analiza:
\begin{enumerate}
    \item \textbf{Potencia específica} del motor ($\text{kW/kg}$).
    \item \textbf{Relación potencia/masa} del vehículo completo.
    \item \textbf{Torque en las ruedas} considerando una reducción típica de $9:1$.
    \item \textbf{Análisis de Desempeño Teórico:} Con base en la relación potencia/masa calculada, investigue si el valor se encuentra dentro del rango típico para vehículos eléctricos compactos de uso urbano. Justifique su respuesta.
\end{enumerate}

\subsection*{Fase 3: Identificación y Resolución de Problemas (10 puntos)}

\subsubsection*{3.1 Detección de Inconsistencias}
Identifica y analiza posibles problemas:
\begin{enumerate}
    \item \textbf{Incompatibilidades} entre especificaciones de diferentes proveedores.
    \item \textbf{Errores potenciales} en conversiones de unidades.
    \item \textbf{Riesgos de seguridad} derivados de especificaciones incorrectas.
\end{enumerate}

\subsubsection*{3.2 Propuestas de Solución}
Desarrolla soluciones técnicas:
\begin{enumerate}
    \item \textbf{Recomendaciones} para resolver inconsistencias encontradas.
    \item \textbf{Especificaciones corregidas} para proveedores.
    \item \textbf{Protocolo de verificación} metrológica para futuros proyectos.
\end{enumerate}

\subsection*{Fase 4: Síntesis y Recomendaciones (10 puntos)}

\subsubsection*{4.1 Informe Ejecutivo}
Elabora un resumen ejecutivo que incluya:
\begin{enumerate}
    \item \textbf{Viabilidad técnica} del proyecto UrbanE.
    \item \textbf{Principales hallazgos} del análisis metrológico.
    \item \textbf{Recomendaciones estratégicas} para la empresa EcoMotion México.
\end{enumerate}

\vspace{9cm}




\section*{Hoja de Trabajo (Sugerida)}

A continuación, se presenta una tabla sugerida para organizar los resultados obtenidos en cada fase del caso de estudio. Utilizarla facilitará la estructuración de su reporte.

\subsection*{Fase 1: Análisis y Conversión de Unidades}

\begin{tabular}{|l|c|c|c|}
\hline
\textbf{Concepto} & \textbf{Valor Original} & \textbf{Resultado (SI)} & \textbf{Unidades} \\
\hline
Longitud total del chasis & $12.5 \, \text{ft}$ & & m \\
Ancho del chasis & $5.8 \, \text{ft}$ & & m \\
Altura del chasis & $5.2 \, \text{ft}$ & & m \\
Distancia entre ejes & $8.2 \, \text{ft}$ & & m \\
\hline
Peso de la estructura & $1,200 \, \text{lbs}$ & & kg \\
Capacidad de carga & $800 \, \text{lbs}$ & & kg \\
\hline
Presión de neumáticos & $2.2 \, \text{bar}$ & & Pa \\
Presión de neumáticos & $2.2 \, \text{bar}$ & & psi \\
\hline
\end{tabular}

\vspace{5mm}
\textbf{Análisis de Compatibilidad de Neumáticos:}\vspace{2mm}

\textit{¿La carga máxima por neumático es suficiente? (Sí/No y justificación)}

\hrulefill

\vspace{5mm}
\textit{¿La velocidad máxima del neumático es adecuada? (Sí/No y justificación)}

\hrulefill

\subsection*{Fase 2: Cálculos de Rendimiento y Eficiencia}

\begin{tabular}{|l|c|c|}
\hline
\textbf{Cálculo Realizado} & \textbf{Resultado} & \textbf{Unidades} \\
\hline
Densidad energética de la batería & & Wh/kg \\
Consumo energético estimado & & kWh/100km \\
Autonomía teórica & & km \\
\hline
Tiempo de carga (3.3 kW) & & horas \\
Tiempo de carga (7.4 kW) & & horas \\
Tiempo de carga (22 kW) & & horas \\
\hline
Potencia específica del motor & & kW/kg \\
Relación potencia/masa & & W/kg \\
Torque en las ruedas & & N m \\
\hline
\end{tabular}

\vspace{5mm}

\textbf{Análisis de Desempeño Teórico:}\vspace{2mm}

\textit{Con base en la relación potencia/masa, ¿el valor es típico para un vehículo urbano? Justifique.}

\hrulefill

\subsection*{Fase 3 y 4: Análisis y Recomendaciones}

\subsubsection*{3.1 Incompatibilidades y Riesgos Detectados}
\textit{Enumere y describa las incompatibilidades o riesgos encontrados.}

\hrulefill

\subsubsection*{3.2 Propuestas de Solución}
\textit{Describa las recomendaciones técnicas para resolver las inconsistencias.}

\hrulefill

\subsubsection*{4.1 Informe Ejecutivo}
\textit{Redacte aquí el resumen ejecutivo con la viabilidad, hallazgos y recomendaciones.}

%\vspace{4cm}
\hrulefill

\section*{Forma de Entrega del Trabajo}

El estudiante deberá entregar un \hl{único archivo en formato PDF} (`.pdf`) que corresponde al reporte técnico del caso de estudio. El documento debe ser claro, ordenado y contener los siguientes elementos:

\subsection*{1. Datos Generales (Portada)}
\begin{itemize}
    \item Nombre completo del estudiante y matrícula.
    \item Asignatura: Metrología.
    \item Nombre de la actividad: `AVS.01.01.01: Análisis Metrológico de un Vehículo Eléctrico Compacto`.
    \item Nombre del docente.
    \item Fecha de elaboración.
\end{itemize}

\subsection*{2. Cuestionario Previo}
Respuestas claras y bien fundamentadas a las 5 preguntas de la sección ``ACTIVIDAD PREVIA AL DESARROLLO DEL CASO''.

\subsection*{3. Desarrollo del Caso de Estudio}
\begin{itemize}
    \item Fase 1: Análisis y Conversión de Unidades.
    \item Fase 2: Cálculos de Rendimiento y Eficiencia.
    \item Fase 3: Identificación y Resolución de Problemas.
    \item Fase 4: Síntesis y Recomendaciones.
\end{itemize}

\subsection*{4. Conclusiones y Bibliografía}
\begin{itemize}
    \item \textbf{Conclusiones:} Reflexión personal y técnica sobre la importancia de la metrología en el proyecto (mínimo 5 renglones).
    \item \textbf{Bibliografía:} Incluir al menos dos fuentes de consulta en formato APA.
\end{itemize}

\section*{Recursos de Apoyo}

\subsection*{Tablas de Conversión Requeridas}
\begin{itemize}
    \item Sistema Imperial a Sistema Internacional
    \item Múltiplos y submúltiplos del SI
    \item Conversiones energéticas ($\text{kWh}$, $\text{J}$, $\text{cal}$)
    \item Conversiones de presión ($\text{bar}$, $\text{Pa}$, $\text{psi}$)
\end{itemize}

\subsection*{Fórmulas Básicas}
\begin{itemize}
    \item Densidad energética: $E/m$ ($\text{Wh/kg}$)
    \item Potencia específica: $P/m$ ($\text{kW/kg}$)
    \item Eficiencia: $(\text{Energía útil} / \text{Energía total}) \times 100\%$
    \item Autonomía teórica: Capacidad batería / Consumo promedio
\end{itemize}

\subsection*{Referencias Técnicas Sugeridas}
\begin{itemize}
    \item Normas mexicanas NOM aplicables a vehículos eléctricos
    \item Estándares internacionales ISO para metrología automotriz
    \item Especificaciones técnicas de vehículos eléctricos comerciales
\end{itemize}

\end{document}