% --- Lectura de Apoyo: Ciclos Termodinámicos en Sistemas Automotrices ---
\documentclass{article}

% --- Cargar la plantilla de estilo UTP ---
\usepackage{utp-doc}
\usepackage{tikz}
\usepackage{amsmath}
\usepackage{amsfonts}
\usepackage{pgfplots}
\usepackage{booktabs}
\usepackage{multirow}
\usepackage{array}

% --- Configuración de pgfplots ---
\pgfplotsset{compat=1.16}

\begin{document}

% --- Usar el comando personalizado para el título ---
\practicatitle{Lectura de Apoyo: ``Ciclos Termodinámicos en Sistemas Automotrices''}

Esta lectura de apoyo complementa las sesiones presenciales del tema de \hl{Ciclos Termodinámicos} correspondiente a la \textbf{Unidad IV} de la asignatura E-TEA-3. Su objetivo es proporcionar bases teóricas sólidas para comprender el funcionamiento de \hl{motores de combustión interna} y su \hl{análisis termodinámico}.

\vspace{5mm}

\section*{INTRODUCCIÓN}

Los \hl{ciclos termodinámicos} constituyen la base fundamental para comprender el funcionamiento de los \hl{motores de combustión interna} utilizados en la industria automotriz. Estos ciclos describen las transformaciones energéticas que ocurren en el interior de los motores, permitiendo convertir la \hl{energía química} del combustible en \hl{trabajo mecánico útil}.

En el contexto de la \hl{ingeniería automotriz}, el dominio de estos conceptos es esencial para:
\begin{itemize}
    \item Optimizar el \hl{rendimiento} de los motores
    \item Reducir el \hl{consumo de combustible}
    \item Minimizar las \hl{emisiones contaminantes}
    \item Mejorar la \hl{durabilidad} de los componentes
\end{itemize}

\vspace{5mm}

\section*{CONCEPTOS FUNDAMENTALES}

\subsection*{¿Qué es un Ciclo Termodinámico?}

Un \hl{ciclo termodinámico} es una secuencia de \textbf{procesos} que experimenta un fluido de trabajo (generalmente aire o mezcla aire-combustible) y que lo lleva desde un estado inicial, a través de varios estados intermedios, hasta regresar exactamente al estado inicial.

\textbf{Componentes esenciales de un ciclo:}
\begin{itemize}
    \item \textbf{Fluido de trabajo:} Aire o mezcla aire-combustible que experimenta las transformaciones
    \item \textbf{Fuente de calor:} Proceso de combustión que aporta energía térmica
    \item \textbf{Sumidero de calor:} Ambiente que recibe la energía térmica no convertida en trabajo
    \item \textbf{Dispositivo de trabajo:} Pistón o turbina que convierte la energía térmica en trabajo mecánico
\end{itemize}

\subsection*{Parámetros Fundamentales}

\textbf{Eficiencia Térmica:}
La eficiencia térmica ($\eta$) de un ciclo termodinámico se define como la fracción del calor suministrado que se convierte en trabajo útil:

$$\eta = \frac{W_{\text{neto}}}{Q_{\text{entrada}}} = \frac{Q_{\text{entrada}} - Q_{\text{salida}}}{Q_{\text{entrada}}}$$

Donde:
\begin{itemize}
    \item $W_{\text{neto}}$ = trabajo neto producido por el ciclo
    \item $Q_{\text{entrada}}$ = calor suministrado al sistema
    \item $Q_{\text{salida}}$ = calor rechazado por el sistema
\end{itemize}

\textbf{Relación de Compresión:}
En motores de pistón, la relación de compresión ($r$) es un parámetro crucial definido como:

$$r = \frac{V_{\text{max}}}{V_{\text{min}}}$$

Donde $V_{\text{max}}$ es el volumen cuando el pistón está en el punto muerto inferior y $V_{\text{min}}$ es el volumen cuando el pistón está en el punto muerto superior.

\vspace{5mm}

\section*{CICLO DE CARNOT: EL MOTOR TÉRMICO IDEAL}

\subsection*{Descripción del Ciclo}

El \hl{ciclo de Carnot}, propuesto por Sadi Carnot en 1824, representa el \textbf{motor térmico más eficiente} teóricamente posible que opera entre dos reservorios térmicos. Aunque es imposible de implementar prácticamente, establece el \hl{límite superior de eficiencia} para cualquier motor térmico.

\textbf{Los cuatro procesos del ciclo de Carnot:}

\begin{enumerate}
    \item \textbf{Proceso 1-2: Expansión isotérmica}
    \begin{itemize}
        \item Temperatura constante ($T_{\text{alta}}$)
        \item El sistema absorbe calor $Q_{\text{entrada}}$ de la fuente caliente
        \item Se produce trabajo durante la expansión
    \end{itemize}
    
    \item \textbf{Proceso 2-3: Expansión adiabática}
    \begin{itemize}
        \item No hay intercambio de calor ($Q = 0$)
        \item La temperatura disminuye de $T_{\text{alta}}$ a $T_{\text{baja}}$
        \item Continúa la expansión y producción de trabajo
    \end{itemize}
    
    \item \textbf{Proceso 3-4: Compresión isotérmica}
    \begin{itemize}
        \item Temperatura constante ($T_{\text{baja}}$)
        \item El sistema rechaza calor $Q_{\text{salida}}$ hacia el sumidero frío
        \item Se requiere trabajo para la compresión
    \end{itemize}
    
    \item \textbf{Proceso 4-1: Compresión adiabática}
    \begin{itemize}
        \item No hay intercambio de calor ($Q = 0$)
        \item La temperatura aumenta de $T_{\text{baja}}$ a $T_{\text{alta}}$
        \item Se completa el ciclo
    \end{itemize}
\end{enumerate}

\subsection*{Eficiencia del Ciclo de Carnot}

La eficiencia del ciclo de Carnot depende únicamente de las temperaturas de los reservorios térmicos:

$$\eta_{\text{Carnot}} = 1 - \frac{T_{\text{fría}}}{T_{\text{caliente}}}$$

\textbf{Ejemplo numérico:}
Para un motor que opera con:
\begin{itemize}
    \item $T_{\text{caliente}} = 800^\circ C = 1073 K$ (temperatura de combustión)
    \item $T_{\text{fría}} = 25^\circ C = 298 K$ (temperatura ambiente)
\end{itemize}

La eficiencia teórica sería:
$$\eta_{\text{Carnot}} = 1 - \frac{298}{1073} = 0.722 = 72.2\%$$

\textbf{Implicaciones para motores automotrices:}
\begin{itemize}
    \item Ningún motor real puede superar esta eficiencia teórica
    \item Los motores reales alcanzan solo 25-35\% de eficiencia
    \item Las pérdidas se deben a fricción, transferencia de calor, combustión incompleta, etc.
\end{itemize}

\vspace{5mm}

\section*{CICLO OTTO: MOTORES DE GASOLINA}

\subsection*{Descripción del Ciclo}

El \hl{ciclo Otto}, desarrollado por Nikolaus Otto en 1876, es la base teórica de los \textbf{motores de gasolina} modernos. Este ciclo también se conoce como \textbf{ciclo de encendido por chispa} y opera con cuatro procesos principales.

\textbf{Los cuatro procesos del ciclo Otto:}

\begin{enumerate}
    \item \textbf{Proceso 1-2: Compresión adiabática}
    \begin{itemize}
        \item Las válvulas están cerradas
        \item La mezcla aire-combustible se comprime
        \item La temperatura y presión aumentan significativamente
        \item Relación de compresión típica: 8:1 a 12:1
    \end{itemize}
    
    \item \textbf{Proceso 2-3: Combustión a volumen constante}
    \begin{itemize}
        \item La bujía produce una chispa que inicia la combustión
        \item Liberación explosiva de energía química
        \item La presión aumenta dramáticamente mientras el volumen permanece constante
        \item Este es el proceso que diferencia al Otto del Carnot
    \end{itemize}
    
    \item \textbf{Proceso 3-4: Expansión adiabática (carrera de trabajo)}
    \begin{itemize}
        \item Los gases calientes empujan el pistón hacia abajo
        \item \textbf{Única carrera que produce trabajo útil}
        \item La temperatura y presión disminuyen durante la expansión
    \end{itemize}
    
    \item \textbf{Proceso 4-1: Escape a volumen constante}
    \begin{itemize}
        \item La válvula de escape se abre
        \item Los gases quemados son expulsados al ambiente
        \item La presión regresa rápidamente a las condiciones iniciales
    \end{itemize}
\end{enumerate}

\subsection*{Eficiencia del Ciclo Otto}

La eficiencia teórica del ciclo Otto depende únicamente de la relación de compresión:

$$\eta_{\text{Otto}} = 1 - \frac{1}{r^{\gamma-1}}$$

Donde:
\begin{itemize}
    \item $r$ = relación de compresión
    \item $\gamma$ = relación de calores específicos ($\approx$ 1.4 para aire)
\end{itemize}

\textbf{Ejemplo de cálculo:}
Para un motor con relación de compresión $r = 10:1$:
$$\eta_{\text{Otto}} = 1 - \frac{1}{10^{0.4}} = 1 - 0.398 = 0.602 = 60.2\%$$

\textbf{Factores que afectan la eficiencia real:}
\begin{itemize}
    \item \textbf{Detonación (knock):} Limita la relación de compresión máxima
    \item \textbf{Pérdidas por fricción:} Entre pistones, anillos y cilindros
    \item \textbf{Transferencia de calor:} Hacia las paredes del cilindro
    \item \textbf{Combustión incompleta:} No todo el combustible se quema eficientemente
    \item \textbf{Tiempo finito de combustión:} La combustión real no es instantánea
\end{itemize}

\subsection*{Aplicaciones Automotrices del Ciclo Otto}

\textbf{Ventajas:}
\begin{itemize}
    \item Alta potencia específica (potencia por unidad de peso)
    \item Funcionamiento suave y silencioso
    \item Arranque fácil en frío
    \item Menores emisiones de NOx comparado con diesel
    \item Tecnología madura y bien establecida
\end{itemize}

\textbf{Aplicaciones típicas:}
\begin{itemize}
    \item Automóviles de pasajeros
    \item Vehículos deportivos
    \item Motocicletas
    \item Equipos de jardinería y herramientas menores
\end{itemize}

\vspace{5mm}

\section*{CICLO DIESEL: MOTORES DE COMPRESIÓN}

\subsection*{Descripción del Ciclo}

El \hl{ciclo Diesel}, desarrollado por Rudolf Diesel en 1892, es la base de los \textbf{motores diesel} modernos. La principal diferencia con el ciclo Otto es que la combustión ocurre a \textbf{presión constante} en lugar de volumen constante.

\textbf{Los cuatro procesos del ciclo Diesel:}

\begin{enumerate}
    \item \textbf{Proceso 1-2: Compresión adiabática}
    \begin{itemize}
        \item \textbf{Solo aire} se comprime (sin combustible)
        \item Relación de compresión muy alta: 14:1 a 22:1
        \item La temperatura final supera los 500$^\circ$C, suficiente para autoencendido
        \item Mayor compresión que en Otto debido a la ausencia de combustible
    \end{itemize}
    
    \item \textbf{Proceso 2-3: Combustión a presión constante}
    \begin{itemize}
        \item El combustible se inyecta durante la expansión inicial
        \item \textbf{Combustión controlada}, no explosiva como en Otto
        \item La temperatura aumenta mientras el volumen aumenta
        \item La presión se mantiene aproximadamente constante
    \end{itemize}
    
    \item \textbf{Proceso 3-4: Expansión adiabática}
    \begin{itemize}
        \item Continuación de la expansión después de completarse la combustión
        \item Extracción del trabajo útil
        \item Disminución de presión y temperatura
    \end{itemize}
    
    \item \textbf{Proceso 4-1: Escape a volumen constante}
    \begin{itemize}
        \item Similar al ciclo Otto
        \item Liberación rápida de gases quemados
    \end{itemize}
\end{enumerate}

\subsection*{Eficiencia del Ciclo Diesel}

La eficiencia del ciclo Diesel depende tanto de la relación de compresión como de la relación de corte:

$$\eta_{\text{Diesel}} = 1 - \frac{1}{r^{\gamma-1}} \times \frac{r_c^\gamma - 1}{\gamma(r_c - 1)}$$

Donde:
\begin{itemize}
    \item $r$ = relación de compresión
    \item $r_c$ = relación de corte ($V_3/V_2$, relacionada con la duración de la combustión)
    \item $\gamma$ = relación de calores específicos
\end{itemize}

\textbf{Ventajas del ciclo Diesel:}
\begin{itemize}
    \item \textbf{Mayor eficiencia térmica:} 40-45\% vs 25-35\% del Otto
    \item \textbf{Mayor densidad energética:} El diesel contiene más energía por litro
    \item \textbf{Mayor durabilidad:} Construcción más robusta
    \item \textbf{Mejor economía de combustible:} Especialmente en aplicaciones de carga
\end{itemize}

\subsection*{Aplicaciones Automotrices del Ciclo Diesel}

\textbf{Características principales:}
\begin{itemize}
    \item Mayor torque a bajas revoluciones
    \item Mejor para aplicaciones de trabajo pesado
    \item Mayor peso del motor debido a la construcción robusta
    \item Emisiones más altas de NOx y partículas
\end{itemize}

\textbf{Aplicaciones típicas:}
\begin{itemize}
    \item Camiones de carga
    \item Autobuses de transporte público
    \item Vehículos comerciales
    \item Maquinaria agrícola e industrial
    \item Generación de energía estacionaria
\end{itemize}

\vspace{5mm}

\section*{ANÁLISIS COMPARATIVO}

\subsection*{Comparación de Eficiencias Teóricas}

La siguiente tabla compara las características principales de los tres ciclos estudiados:

\begin{center}
\begin{tabular}{|p{3cm}|p{3cm}|p{3cm}|p{3cm}|}
\hline
\textbf{Característica} & \textbf{Carnot} & \textbf{Otto} & \textbf{Diesel} \\
\hline\hline
\textbf{Eficiencia teórica} & 72\% (ejemplo) & 60\% (r=10) & 65\% (r=18) \\
\hline
\textbf{Eficiencia real} & Imposible de lograr & 25-35\% & 35-45\% \\
\hline
\textbf{Tipo de combustión} & Isotérmica ideal & Volumen constante & Presión constante \\
\hline
\textbf{Aplicabilidad} & Solo teórica & Motores de gasolina & Motores diesel \\
\hline
\textbf{Relación de compresión} & Variable & 8:1 - 12:1 & 14:1 - 22:1 \\
\hline
\end{tabular}
\end{center}

\subsection*{Criterios de Selección}

\textbf{Ciclo Otto recomendado para:}
\begin{itemize}
    \item Conducción urbana con arranques y paradas frecuentes
    \item Aplicaciones que requieren alta velocidad de rotación
    \item Vehículos ligeros de pasajeros
    \item Situaciones donde el peso del motor es crítico
\end{itemize}

\textbf{Ciclo Diesel recomendado para:}
\begin{itemize}
    \item Transporte de carga pesada
    \item Recorridos de larga distancia
    \item Aplicaciones que requieren alto torque a bajas RPM
    \item Situaciones donde la economía de combustible es prioritaria
\end{itemize}

\vspace{5mm}

\section*{TENDENCIAS FUTURAS}

\subsection*{Innovaciones Tecnológicas}

La industria automotriz está implementando diversas tecnologías para mejorar la eficiencia de los motores de combustión interna:

\textbf{Optimizaciones del ciclo Otto:}
\begin{itemize}
    \item \textbf{Inyección directa:} Mejor control de la mezcla aire-combustible
    \item \textbf{Turbocompresión:} Aumento de la densidad del aire de admisión
    \item \textbf{Relación de compresión variable:} Optimización según condiciones de operación
    \item \textbf{Ciclos Atkinson y Miller:} Variaciones que priorizan la eficiencia sobre la potencia
\end{itemize}

\textbf{Mejoras en motores Diesel:}
\begin{itemize}
    \item \textbf{Common Rail:} Sistema de inyección de alta presión
    \item \textbf{Turbocompresión con intercooler:} Mejor llenado de cilindros
    \item \textbf{Sistemas de postratamiento:} Reducción de emisiones NOx y partículas
    \item \textbf{Diesel limpio:} Combustibles con menor contenido de azufre
\end{itemize}

\subsection*{Hibridización y Electrificación}

La transición hacia la movilidad sostenible está generando:
\begin{itemize}
    \item \textbf{Motores híbridos:} Combinación de ciclos térmicos con propulsión eléctrica
    \item \textbf{Recuperación de energía:} Aprovechamiento del calor de escape
    \item \textbf{Optimización integral:} Gestión inteligente de sistemas de propulsión
\end{itemize}

\vspace{5mm}

\section*{EJERCICIOS DE APLICACIÓN}

\subsection*{Ejercicio 1: Cálculo de Eficiencia Otto}

Un motor de gasolina tiene una relación de compresión de 9:1. Calcular:
\begin{enumerate}
    \item La eficiencia teórica del ciclo Otto
    \item El trabajo neto si el calor de entrada es 2000 kJ
    \item La explicación de por qué la eficiencia real es menor
\end{enumerate}

\textbf{Solución:}
\begin{enumerate}
    \item $\eta_{\text{Otto}} = 1 - \frac{1}{9^{0.4}} = 1 - 0.426 = 0.574 = 57.4\%$
    \item $W_{\text{neto}} = \eta \times Q_{\text{entrada}} = 0.574 \times 2000 = 1148 \text{ kJ}$
    \item La eficiencia real es menor debido a pérdidas por fricción, transferencia de calor, combustión incompleta y limitaciones de tiempo de los procesos reales.
\end{enumerate}

\subsection*{Ejercicio 2: Comparación Otto vs Diesel}

Compare dos motores de la misma potencia nominal:
\begin{itemize}
    \item Motor Otto: r = 10:1, consumo 8 L/100km
    \item Motor Diesel: r = 16:1, consumo 6 L/100km
\end{itemize}

Para un recorrido anual de 20,000 km, determine cuál es más económico considerando precios de combustible de \$20/L para gasolina y \$18/L para diesel.

\vspace{5mm}

\section*{CONCLUSIONES}

El estudio de los \hl{ciclos termodinámicos} es fundamental para comprender el funcionamiento de los motores automotrices y optimizar su desempeño. Los conceptos principales que debe dominar el ingeniero en sistemas automotrices son:

\begin{enumerate}
    \item El \textbf{ciclo de Carnot} establece el límite teórico de eficiencia para cualquier motor térmico
    \item El \textbf{ciclo Otto} es la base de los motores de gasolina, con combustión a volumen constante
    \item El \textbf{ciclo Diesel} opera con combustión a presión constante y ofrece mayor eficiencia
    \item La \textbf{selección del tipo de motor} depende de la aplicación específica y los requerimientos de operación
    \item Las \textbf{innovaciones tecnológicas} continúan mejorando la eficiencia y reduciendo las emisiones
\end{enumerate}

El dominio de estos conceptos permite tomar decisiones informadas en el diseño, selección y optimización de sistemas de propulsión automotriz, contribuyendo a una movilidad más eficiente y sostenible.

\vspace{5mm}

\section*{REFERENCIAS BIBLIOGRÁFICAS}

\begin{itemize}
    \item Çengel, Y. A., \& Boles, M. A. (2019). \textit{Termodinámica}. 8va Edición. McGraw-Hill. Capítulos 9-11.
    \item Moran, M. J., Shapiro, H. N., Boettner, D. D., \& Bailey, M. B. (2018). \textit{Fundamentos de Termodinámica Técnica}. 8va Edición. Reverté. Secciones sobre ciclos de potencia.
    \item Heywood, J. B. (2018). \textit{Internal Combustion Engine Fundamentals}. 2da Edición. McGraw-Hill.
    \item Ferguson, C. R., \& Kirkpatrick, A. T. (2016). \textit{Internal Combustion Engines: Applied Thermosciences}. 3ra Edición. Wiley.
\end{itemize}

\vspace{5mm}

\section*{RECURSOS DIGITALES COMPLEMENTARIOS}

\begin{itemize}
    \item Simuladores interactivos de ciclos termodinámicos: \texttt{PhET Simulations}
    \item Software de cálculo termodinámico: \texttt{CoolProp}, \texttt{REFPROP}
    \item Videos educativos de motores en funcionamiento: Canal YouTube \textit{EngineeringExplained}
    \item Documentación técnica de fabricantes automotrices: Manuales de servicio
\end{itemize}

\end{document}