% --- Lectura de Apoyo: Los Pilares de la Termodinámica ---
\documentclass{article}

% --- Cargar la plantilla de estilo UTP ---
\usepackage{utp-doc}
\usepackage{tikz}
\usepackage{amsmath}
\usepackage{amsfonts}
\usepackage{pgfplots}
\usepackage{booktabs}
\usepackage{multirow}
\usepackage{array}

% --- Configuración de Bibliografía con BibLaTeX y APA 7 ---
\usepackage[style=apa]{biblatex}
\addbibresource{bibliografia.bib}

% --- Configuración de pgfplots ---
\pgfplotsset{compat=1.16}

\begin{document}

% --- Usar el comando personalizado para el título ---
\practicatitle{Lectura de Apoyo: ``Los Pilares de la Termodinámica: Un Viaje por los Procesos Fundamentales''}

Esta lectura de apoyo complementa las sesiones presenciales del tema de \hl{Procesos Termodinámicos} correspondiente a la \textbf{Unidad IV} de la asignatura E-TEA-3. Su objetivo es proporcionar una comprensión sólida de las \hl{transformaciones fundamentales de energía} que constituyen los bloques básicos de construcción de los ciclos termodinámicos.

\vspace{5mm}

\section*{INTRODUCCIÓN: LOS BLOQUES DE CONSTRUCCIÓN DE LAS MÁQUINAS TÉRMICAS}

¡Hola de nuevo! En nuestro viaje anterior, exploramos el Ciclo de Carnot, el modelo ideal de eficiencia. Ahora, vamos a dar un paso atrás para entender sus componentes básicos: los \hl{procesos termodinámicos}.

Un \hl{proceso termodinámico} es cualquier transformación en la que un sistema (como un gas en un pistón) cambia de un estado a otro \parencite{cengel2002}. Imagina que son los ``movimientos'' individuales que, combinados, crean el ``baile'' completo de un ciclo de motor o un refrigerador.

Nos centraremos en cuatro procesos idealizados que son la base de casi todos los ciclos termodinámicos: \textbf{isotérmico}, \textbf{isobárico}, \textbf{isocórico} y \textbf{adiabático}. ¡Comencemos!

\vspace{5mm}

\section*{1. PROCESO ISOTÉRMICO: A TEMPERATURA CONSTANTE}

Un \hl{proceso isotérmico} es aquel que ocurre manteniendo la temperatura constante ($T = \text{constante}$).

\subsection*{¿Qué sucede?}

Para que la temperatura no cambie mientras el gas se expande (y realiza trabajo) o se comprime (y se le aplica trabajo), el sistema debe estar en perfecto contacto con un \hl{depósito de calor} (un foco caliente o frío).

\textbf{En expansión:} El gas tiende a enfriarse al expandirse. Para mantener la temperatura, absorbe calor ($Q$) del depósito.

\textbf{En compresión:} El gas tiende a calentarse al ser comprimido. Para mantener la temperatura, cede calor ($Q$) al depósito.

\subsection*{La Clave es la Lentitud}

Este proceso debe ser muy lento (\hl{cuasiestático}) para que el calor tenga tiempo de fluir y mantener la temperatura uniforme y constante.

\subsection*{Primera Ley de la Termodinámica}

$\Delta U = Q - W$: Para un gas ideal, la energía interna ($\Delta U$) solo depende de la temperatura \parencite{wark2001}. Si la temperatura es constante, $\Delta U = 0$. Por lo tanto, la ley se simplifica a:

$$Q = W$$

Esto significa que todo el calor que entra se convierte en trabajo de expansión, y todo el trabajo de compresión se convierte en calor que sale.

\subsection*{Diagrama Presión-Volumen (P-V)}

Se representa como una curva suave llamada \hl{hipérbola}. A medida que el volumen aumenta, la presión disminuye proporcionalmente.

\vspace{5mm}

\section*{2. PROCESO ISOBÁRICO: A PRESIÓN CONSTANTE}

Un \hl{proceso isobárico} es aquel que ocurre manteniendo la presión constante ($P = \text{constante}$).

\subsection*{¿Qué sucede?}

Imagina un pistón libre de moverse pero con un peso constante encima. Si calentamos el gas, este se expandirá y levantará el pistón (y el peso), pero la presión ejercida por el peso y la atmósfera no cambiará.

\begin{itemize}
    \item Si se añade calor ($Q$), el gas se expande ($\Delta V$ es positivo) y realiza trabajo ($W$). Su temperatura y energía interna también aumentan.
    \item Si se retira calor, el gas se contrae y su temperatura y energía interna disminuyen.
\end{itemize}

\subsection*{Primera Ley de la Termodinámica}

Aquí, ningún término es necesariamente cero. El calor añadido se reparte entre aumentar la energía interna y realizar trabajo:

$$\Delta U = Q - W$$

El trabajo es fácil de calcular, ya que la presión es constante:

$$W = P \cdot \Delta V = P \cdot (V_{\text{final}} - V_{\text{inicial}})$$

\subsection*{Diagrama P-V}

Se representa como una \hl{línea horizontal}. El volumen cambia, pero la presión se mantiene en el mismo nivel.

\vspace{5mm}

\section*{3. PROCESO ISOCÓRICO: A VOLUMEN CONSTANTE}

Un \hl{proceso isocórico} (o isométrico) es aquel que ocurre manteniendo el volumen constante ($V = \text{constante}$).

\subsection*{¿Qué sucede?}

El sistema está en un contenedor rígido. Como las paredes no se mueven, el gas no puede expandirse ni comprimirse. Por lo tanto, no se realiza trabajo.

\begin{itemize}
    \item Si se añade calor ($Q$), toda esa energía se invierte en aumentar la energía interna del gas ($\Delta U$), lo que se traduce en un aumento de su temperatura y presión.
    \item Si se retira calor, la energía interna, la temperatura y la presión disminuyen.
\end{itemize}

\subsection*{Primera Ley de la Termodinámica}

Como el cambio de volumen es cero, el trabajo ($W = P \cdot \Delta V$) también es cero \parencite{manrique2001}. La ley se simplifica drásticamente:

$$\Delta U = Q$$

Todo el calor transferido afecta directamente a la energía interna del sistema.

\subsection*{Diagrama P-V}

Se representa como una \hl{línea vertical}. La presión cambia, pero el volumen permanece fijo.

\vspace{5mm}

\section*{4. PROCESO ADIABÁTICO: SIN INTERCAMBIO DE CALOR}

Un \hl{proceso adiabático} es aquel en el que no hay transferencia de calor entre el sistema y su entorno ($Q = 0$) \parencite{wiki:proceso_adiabatico}.

\subsection*{¿Qué sucede?}

El sistema está perfectamente aislado. Cualquier cambio en su energía se debe únicamente al trabajo.

\textbf{Expansión adiabática:} Si el gas se expande, realiza trabajo utilizando su propia energía interna. Como resultado, el gas se enfría. (Ej: el aire que sale de una lata de aire comprimido se siente frío).

\textbf{Compresión adiabática:} Si se realiza trabajo sobre el gas para comprimirlo, esa energía se almacena como energía interna. Como resultado, el gas se calienta. (Ej: el bombín de una bicicleta se calienta al inflar una rueda rápidamente).

\subsection*{La Clave es la Rapidez}

A diferencia del proceso isotérmico, un proceso adiabático real suele ser muy rápido para que el calor no tenga tiempo de escapar.

\subsection*{Primera Ley de la Termodinámica}

Con $Q = 0$, la ley se simplifica a:

$$\Delta U = -W$$

El cambio en la energía interna es el negativo del trabajo realizado.

\subsection*{Diagrama P-V}

Se representa como una curva más pronunciada que una isoterma. Para la misma expansión, la presión en un proceso adiabático cae más rápido porque la temperatura también está bajando.

\vspace{5mm}

\section*{TABLA RESUMEN DE LOS PROCESOS TERMODINÁMICOS}

\begin{center}
\begin{tabular}{|p{2.5cm}|p{2.5cm}|p{2.5cm}|p{2.5cm}|p{2.5cm}|}
\hline
\textbf{Proceso} & \textbf{Variable Constante} & \textbf{Transferencia de Calor (Q)} & \textbf{Trabajo Realizado (W)} & \textbf{Cambio de Energía Interna ($\Delta U$)} \\
\hline\hline
\textbf{Isotérmico} & Temperatura ($T$) & $Q = W$ & Varía & $\Delta U = 0$ \\
\hline
\textbf{Isobárico} & Presión ($P$) & Varía & $W = P \cdot \Delta V$ & $\Delta U = Q - W$ \\
\hline
\textbf{Isocórico} & Volumen ($V$) & $Q = \Delta U$ & $W = 0$ & Varía \\
\hline
\textbf{Adiabático} & Sin calor ($Q = 0$) & $Q = 0$ & $W = -\Delta U$ & Varía \\
\hline
\end{tabular}
\end{center}

\begin{center}
\begin{tabular}{|p{2.5cm}|p{4cm}|}
\hline
\textbf{Proceso} & \textbf{Apariencia en Diagrama P-V} \\
\hline\hline
\textbf{Isotérmico} & Curva Hiperbólica \\
\hline
\textbf{Isobárico} & Línea Horizontal \\
\hline
\textbf{Isocórico} & Línea Vertical \\
\hline
\textbf{Adiabático} & Curva Pronunciada \\
\hline
\end{tabular}
\end{center}

\vspace{5mm}

\section*{CONCLUSIÓN: EL ALFABETO DE LA ENERGÍA}

Estos cuatro procesos son como el \hl{alfabeto de la termodinámica} \parencite{wiki:termodinamica}. Al combinarlos de diferentes maneras, podemos describir y analizar los ciclos que impulsan nuestro mundo, desde los \textbf{motores de los coches} (ciclos Otto y Diesel) hasta las \textbf{centrales eléctricas} (ciclo Rankine) y los \textbf{sistemas de refrigeración}. 

Comprender cómo se comporta la energía en cada uno de estos pasos es fundamental para diseñar sistemas más eficientes y sostenibles.

La comprensión profunda de estos procesos fundamentales permite:

\begin{itemize}
    \item Analizar el comportamiento de sistemas térmicos complejos
    \item Optimizar el diseño de motores y máquinas térmicas
    \item Predecir el rendimiento de ciclos termodinámicos
    \item Desarrollar tecnologías más eficientes y ambientalmente responsables
\end{itemize}

En las siguientes sesiones, aplicaremos estos conceptos fundamentales al análisis detallado de ciclos termodinámicos específicos utilizados en sistemas automotrices.

\vspace{10mm}
\hrule

\section*{Referencias}
\printbibliography

\end{document}