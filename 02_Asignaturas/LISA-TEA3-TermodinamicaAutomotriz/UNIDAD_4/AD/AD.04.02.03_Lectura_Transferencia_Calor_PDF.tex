\documentclass{article}
\usepackage{graphicx}      % Para imágenes
\usepackage[svgnames]{xcolor} % Para colores avanzados (permite nombres como 'SeaGreen')
\usepackage{geometry}      % Para controlar los márgenes de la página
\usepackage{fancyhdr}      % Para personalizar encabezados y pies de página
\usepackage{titlesec}      % Para personalizar los títulos de las secciones
\usepackage[utf8]{inputenc}
\usepackage[spanish]{babel}
% --- Configuración de Fuente Principal (Estilo Helveca) ---
\usepackage{helvet} % 
\usepackage{amsmath}
\usepackage{xcolor}
\usepackage{tikz}
\usepackage{geometry}
\usepackage{pifont}
\usepackage{float}
\usepackage{setspace}
\usepackage{utp-doc}
% Configuración de página
%\geometry{a4paper, margin=2.5cm}
%\renewcommand{\familydefault}{\sfdefault}
% Colores institucionales
%\definecolor{UTPGreen}{HTML}{00833D}
%\definecolor{UTPGray}{HTML}{666666}
% Configuración de encabezados
%\pagestyle{fancy}
%\fancyhf{}
%\rhead{\thepage}
%\lhead{\textcolor{UTPGray}{\small AD.04.02.03 - Transferencia de Calor}}
\begin{document}
	
	% Establecer espaciado entre líneas a 1.5
	\onehalfspacing
	
	%
	% --- Título de la Lectura ---
	\practicatitle{AD.04.02.03: Lectura de Transferencia de Calor - Conceptos Fundamentales y Mecanismos}
	
	% --- Metadatos de la Actividad ---
	\textbf{Asignatura:} Termodinámica Automotriz \\
	\textbf{Unidad IV:} - Sistemas y Ciclos de Potencia de Gas
	
	\vspace{5mm}
	\hrule
	\vspace{5mm}

\vspace{1cm}

    % --- Contenido ---
    \tableofcontents
    \newpage

    \section{Introducción}

    La \hl{transferencia de calor} es una disciplina fundamental en la ingeniería que se ocupa de la \hl{rapidez o razón de la transferencia de energía} como calor. A menudo se confunde con la \hl{termodinámica}, pero es crucial entender su distinción para el desarrollo de competencias en \hl{ingeniería automotriz}.

    En el contexto automotriz, la \hl{gestión térmica} es esencial para el rendimiento óptimo de sistemas como \hl{motores de combustión interna}, \hl{sistemas de enfriamiento}, \hl{frenos}, \hl{sistemas de climatización} y \hl{componentes electrónicos}. El dominio de los principios de transferencia de calor permite a los ingenieros diseñar sistemas más \hl{eficientes}, \hl{duraderos} y \hl{ambientalmente responsables}.

    Esta lectura proporcionará una comprensión integral de los \hl{conceptos fundamentales} de transferencia de calor, los \hl{tres mecanismos básicos} (conducción, convección y radiación), y sus \hl{aplicaciones prácticas} en la ingeniería automotriz, preparando al estudiante para enfrentar desafíos reales en el análisis y diseño de sistemas térmicos.

    \section{Desarrollo}

    \subsection{Termodinámica vs. Transferencia de Calor}

    La distinción entre \hl{termodinámica} y \hl{transferencia de calor} es fundamental para comprender cómo abordar los problemas térmicos en ingeniería:

    \textbf{Termodinámica}: Esta ciencia se enfoca en la \hl{cantidad de transferencia de calor} que ocurre cuando un sistema pasa de un estado de equilibrio a otro. Por ejemplo, puede determinar cuánto calor debe transferirse para que el café en un termo se enfríe de 90°C a 80°C. Sin embargo, la termodinámica \textbf{no indica cuánto tiempo durará ese proceso}. Se interesa en los \hl{estados de equilibrio} y los cambios entre ellos.

    \textbf{Transferencia de Calor}: A diferencia de la termodinámica, la transferencia de calor se interesa en la \hl{rapidez o razón (por unidad de tiempo)} a la que esta energía se transfiere. Su estudio es esencial en el diseño y evaluación de equipos de ingeniería, como \hl{intercambiadores de calor}, \hl{calderas}, \hl{radiadores} y \hl{sistemas de aire acondicionado}. La transferencia de calor se ocupa de sistemas donde existe un \hl{desequilibrio térmico}.

    El requisito básico para la transferencia de calor es la presencia de una \hl{diferencia de temperatura}. La energía siempre se transfiere del medio con \hl{temperatura más elevada al de temperatura más baja}, y este proceso se detiene cuando ambos alcanzan la misma temperatura. Cuanto mayor sea el \hl{gradiente de temperatura} (diferencia de temperatura por unidad de longitud), mayor será la razón de transferencia de calor.

    \subsection{Calor y Otras Formas de Energía}

    La energía existe en diversas formas, como térmica, mecánica, cinética, potencial, eléctrica, magnética, química y nuclear, cuya suma constituye la \hl{energía total (E)} de un sistema.

    \textbf{Energía Interna (U)}: Es la suma de todas las formas microscópicas de energía relacionadas con la estructura molecular y el grado de actividad molecular de un sistema.

    \begin{itemize}
        \item \textbf{Energía Sensible (o calor sensible)}: La parte de la energía interna asociada con la energía cinética de las moléculas. Es proporcional a la temperatura; a mayor temperatura, mayor energía cinética y, por ende, mayor \hl{energía sensible}.
        
        \item \textbf{Energía Latente (o calor latente)}: La energía interna asociada con la fase de un sistema. Se agrega energía para vencer las fuerzas moleculares durante un \hl{cambio de fase} (por ejemplo, de sólido a gas), lo que eleva la energía interna del sistema en la nueva fase.
        
        \item \textbf{Energía Química (o de enlace)}: Energía interna asociada con los \hl{enlaces atómicos} en una molécula.
        
        \item \textbf{Energía Nuclear}: Energía interna asociada con los enlaces dentro del \hl{núcleo del átomo}.
    \end{itemize}

    En la vida diaria, las formas sensible y latente de la energía interna a menudo se denominan simplemente ``calor''. Sin embargo, en termodinámica se prefiere el término \hl{energía térmica} para evitar confusiones con la ``transferencia de calor''. No obstante, por convención, a la energía térmica se le llama \hl{calor} y a su transferencia se le llama \hl{transferencia de calor}.

    \textbf{Notación y unidades importantes}:
    \begin{itemize}
        \item La \hl{cantidad de calor transferido} se denota por \textbf{Q} (J o kJ)
        \item La \hl{razón de transferencia de calor} (cantidad de calor transferido por unidad de tiempo) se denota por $\dot{Q}$ y tiene unidades de J/s o W. Si $\dot{Q}$ es constante, la cantidad total de calor transferido en un intervalo de tiempo $\Delta t$ es $Q = \dot{Q} \Delta t$
        \item El \hl{flujo de calor} ($\dot{q}$) es la razón de transferencia de calor por unidad de área perpendicular a la dirección de transferencia, expresado como $\dot{q} = \dot{Q}/A$ (W/m²)
    \end{itemize}

    \subsection{Primera Ley de la Termodinámica (Balance de Energía)}

    La primera ley de la termodinámica, también conocida como el \hl{principio de conservación de la energía}, establece que la energía no se crea ni se destruye, solo cambia de forma. Para cualquier sistema que experimenta un proceso, el \hl{cambio neto en la energía total del sistema es igual a la diferencia entre la energía total que entra y la que sale}.

    \begin{equation}
    E_{entrada} - E_{salida} = \Delta E_{sistema}
    \end{equation}

    En forma de razones (por unidad de tiempo): 
    \begin{equation}
    \dot{E}_{entrada} - \dot{E}_{salida} = \frac{dE_{sistema}}{dt}
    \end{equation}

    Para sistemas en \hl{estado estacionario}, donde el estado del sistema no cambia con el tiempo ($\Delta E_{sistema} = 0$), el balance de energía se reduce a: 
    \begin{equation}
    \dot{E}_{entrada} = \dot{E}_{salida}
    \end{equation}

    En el análisis de transferencia de calor, el interés se centra en la \hl{energía térmica}. Si se considera la conversión de otras energías (nuclear, química, mecánica, eléctrica) en energía térmica como \hl{generación de calor} ($E_{gen}$), el balance de energía puede expresarse como:

    \begin{equation}
    Q_{entrada} - Q_{salida} + E_{gen} = \Delta E_{termica,sistema}
    \end{equation}

    \subsubsection{Aplicación del balance de energía}

    \textbf{Sistemas cerrados (masa fija)}: Para la mayoría de los sistemas prácticos, la energía total se reduce a la \hl{energía interna (U)}. Si el sistema es estacionario y solo hay transferencia de calor sin interacción de trabajo, el balance se simplifica a $Q = mc_v \Delta T$.

    \textbf{Sistemas de flujo estacionario}: Muchos aparatos de ingeniería (calentadores de agua, \hl{radiadores de automóviles}) involucran flujo de masa. En condiciones estacionarias, el contenido total de energía del \hl{volumen de control} permanece constante ($\Delta E_{vc} = 0$). Si los cambios en las energías cinética y potencial son despreciables y no hay interacción de trabajo, el balance de energía se reduce a $\dot{Q} = \dot{m} \Delta h = \dot{m} c_p \Delta T$.

    \begin{itemize}
        \item El \hl{gasto de masa} ($\dot{m}$) es la cantidad de masa que fluye por unidad de tiempo. Para flujo unidimensional en un tubo, $\dot{m} = \rho V A_c$ (kg/s).
        \item La \hl{entalpía} (h) es una combinación de energía interna (u) y energía de flujo (Pv), útil para el análisis de fluidos en movimiento: $h = u + Pv$.
    \end{itemize}

    \textbf{Balance de energía en la superficie}: Una superficie no contiene volumen ni masa, por lo tanto, no almacena energía ni genera calor. Para una superficie, el balance de energía es $\dot{E}_{entrada} = \dot{E}_{salida}$. Esto es válido tanto para condiciones estacionarias como transitorias.

    \subsection{Mecanismos Básicos de Transferencia de Calor}

    El calor se puede transferir de tres modos diferentes: \hl{conducción}, \hl{convección} y \hl{radiación}. Todos requieren una diferencia de temperatura y siempre ocurren del medio de mayor temperatura al de menor temperatura.

    \subsubsection{A. Conducción}

    La \hl{conducción} es la \hl{transferencia de energía de las partículas más energéticas de una sustancia a las adyacentes menos energéticas}, como resultado de la interacción entre ellas.

    \textbf{¿Dónde ocurre?}: Puede tener lugar en sólidos, líquidos o gases.
    \begin{itemize}
        \item En \hl{gases y líquidos}, se debe a las colisiones y la difusión de moléculas durante su movimiento aleatorio.
        \item En \hl{sólidos}, se debe a la combinación de vibraciones de las moléculas en una retícula cristalina y al transporte de energía por parte de los \hl{electrones libres}.
    \end{itemize}

    \textbf{Factores que influyen}: La razón de la conducción de calor depende de la \hl{configuración geométrica}, el \hl{espesor del medio}, el \hl{material} y la \hl{diferencia de temperatura}.

    \paragraph{Ley de Fourier de la Conducción del Calor}

    La razón de la conducción de calor a través de una capa plana es proporcional a la diferencia de temperatura y al área de transferencia, e inversamente proporcional al espesor:

    \begin{equation}
    \dot{Q}_{cond} = -kA \frac{dT}{dx} \quad \text{(W)}
    \end{equation}

    Donde:
    \begin{itemize}
        \item \textbf{k} es la \hl{conductividad térmica} del material (W/m·°C o W/m·K)
        \item \textbf{A} es el área de transferencia de calor, siempre normal a la dirección de transferencia
        \item $\frac{dT}{dx}$ es el \hl{gradiente de temperatura}. El signo negativo asegura que la transferencia de calor en la dirección x positiva sea una cantidad positiva, ya que el calor fluye en la dirección de la temperatura decreciente
    \end{itemize}

    \paragraph{Conductividad Térmica (k)}

    Es una medida de la capacidad de un material para conducir calor:

    \begin{itemize}
        \item Un \hl{valor elevado de k} indica un buen conductor de calor (ej. cobre, plata). Estos también suelen ser buenos \hl{conductores eléctricos}.
        \item Un \hl{valor bajo de k} indica un mal conductor o un \hl{aislante} (ej. caucho, madera, aire).
        \item Los \hl{gases puros} tienen baja conductividad térmica; aumenta con la temperatura y disminuye con la masa molar.
        \item Las \hl{conductividades térmicas de los líquidos} suelen estar entre las de los sólidos y los gases, y a menudo decrecen al aumentar la temperatura (el agua es una excepción notable).
        \item En \hl{sólidos}, las conductividades térmicas de los metales puros son elevadas debido a los \hl{electrones libres}. Las \hl{aleaciones metálicas} suelen tener conductividades térmicas mucho más bajas que los metales que las componen.
        \item La conductividad térmica de los materiales \hl{varía con la temperatura}, aunque a menudo se evalúa a una temperatura promedio y se considera constante para simplificar los cálculos.
        \item Un material se considera \hl{isotrópico} si tiene propiedades uniformes en todas las direcciones.
    \end{itemize}

    \paragraph{Difusividad Térmica ($\alpha$)}

    Representa cuán rápido se difunde el calor a través de un material:

    \begin{equation}
    \alpha = \frac{k}{\rho c_p} \quad \text{(m²/s)}
    \end{equation}

    Donde $\rho c_p$ es la \hl{capacidad calorífica} (energía que almacena un material por unidad de volumen).

    Una \hl{alta difusividad térmica} indica una rápida propagación del calor.

    \subsubsection{B. Convección}

    La \hl{convección} es el modo de transferencia de energía entre una \hl{superficie sólida y el líquido o gas adyacente que está en movimiento}, e incluye los efectos combinados de la \hl{conducción y el movimiento del fluido}. Cuanto más rápido se mueve el fluido, mayor es la transferencia de calor por convección.

    \paragraph{Tipos de Convección}

    \textbf{Convección Forzada}: El fluido es forzado a fluir sobre la superficie por medios externos, como un \hl{ventilador}, una \hl{bomba} o el \hl{viento}.

    \textbf{Convección Natural (o Libre)}: El movimiento del fluido es causado por \hl{fuerzas de empuje} inducidas por diferencias de densidad, las cuales son resultado de variaciones de temperatura dentro del fluido.

    \textbf{Convección con Cambio de Fase}: Procesos como la \hl{ebullición} o la \hl{condensación} también se consideran convección debido al movimiento del fluido inducido durante el cambio de fase (ej. burbujas de vapor subiendo).

    \paragraph{Ley de Newton del Enfriamiento}

    La razón de la transferencia de calor por convección es proporcional a la diferencia de temperatura:

    \begin{equation}
    \dot{Q}_{conv} = h A_s (T_s - T_\infty) \quad \text{(W)}
    \end{equation}

    Donde:
    \begin{itemize}
        \item \textbf{h} es el \hl{coeficiente de transferencia de calor por convección} (W/m²·°C o W/m²·K). No es una propiedad del fluido, sino un parámetro experimental que depende de la geometría de la superficie, la naturaleza y velocidad del movimiento del fluido, y sus propiedades.
        \item $A_s$ es el área superficial a través de la cual ocurre la transferencia de calor por convección
        \item $T_s$ es la temperatura de la superficie
        \item $T_\infty$ es la temperatura del fluido suficientemente alejado de la superficie
    \end{itemize}

    \subsubsection{C. Radiación}

    La \hl{radiación} es la \hl{energía emitida por la materia en forma de ondas electromagnéticas (o fotones)}, resultado de cambios en las configuraciones electrónicas de átomos o moléculas.

    \paragraph{Características de la radiación}

    \begin{itemize}
        \item \textbf{No requiere un medio}: A diferencia de la conducción y la convección, la radiación puede transferir calor a través del \hl{vacío} y es el modo más rápido de transferencia (a la velocidad de la luz). Un ejemplo es la energía del Sol que llega a la Tierra.
        
        \item \textbf{Radiación Térmica}: Es la forma de radiación emitida por los cuerpos debido a su temperatura. Todos los cuerpos a una temperatura superior al \hl{cero absoluto} emiten radiación térmica.
    \end{itemize}

    \paragraph{Ley de Stefan-Boltzmann}

    Expresa la razón máxima de radiación que puede emitirse desde una superficie a una temperatura termodinámica $T_s$ (en K o R):

    \begin{equation}
    \dot{Q}_{emitida,max} = \sigma A_s T_s^4 \quad \text{(W)}
    \end{equation}

    Donde $\sigma$ es la \hl{constante de Stefan-Boltzmann} $(5.67 \times 10^{-8} W/m^2·K^4)$.

    La superficie idealizada que emite radiación a esta razón máxima se llama \hl{cuerpo negro}.

    \paragraph{Propiedades Radiativas}

    \textbf{Emisividad ($\varepsilon$)}: La radiación emitida por las superficies reales es menor que la de un cuerpo negro a la misma temperatura. La emisividad $(0 \leq \varepsilon \leq 1)$ es una medida de cuán cerca está una superficie de ser un cuerpo negro (para un cuerpo negro, $\varepsilon = 1$).

    \begin{equation}
    \dot{Q}_{emitida} = \varepsilon A_s T_s^4 \quad \text{(W)}
    \end{equation}

    \textbf{Absortividad ($\alpha$)}: Es la fracción de la energía de radiación incidente sobre una superficie que es absorbida por ella $(0 \leq \alpha \leq 1)$. Un cuerpo negro es un \hl{absorbente perfecto} $(\alpha = 1)$.

    \textbf{Ley de Kirchhoff de la Radiación}: Establece que la emisividad y la absortividad de una superficie a una temperatura y longitud de onda dadas son \hl{iguales}.

    \paragraph{Radiación Neta}

    La diferencia entre la radiación emitida por la superficie y la radiación absorbida. Si una superficie de emisividad $\varepsilon$ y área $A_s$ a temperatura $T_s$ está rodeada por una superficie mucho más grande a temperatura $T_{alred}$, la razón neta de transferencia de calor por radiación es:

    \begin{equation}
    \dot{Q}_{rad} = \varepsilon A_s (T_s^4 - T_{alred}^4) \quad \text{(W)}
    \end{equation}

    Es importante usar \hl{temperaturas termodinámicas (absolutas)} en los cálculos de radiación.

    \textbf{Coeficiente Combinado de Transferencia de Calor} ($h_{combinado}$): Se define para incluir los efectos tanto de la convección como de la radiación, especialmente cuando la radiación es significativa (ej. en \hl{convección natural}), permitiendo expresar la razón total de transferencia de calor como $\dot{Q}_{total} = h_{combinado} A_s (T_s - T_\infty)$.

    \subsection{Mecanismos Simultáneos de Transferencia de Calor}

    Aunque existen tres mecanismos, no siempre los tres pueden ocurrir simultáneamente en un mismo medio:

    \begin{itemize}
        \item \textbf{Sólidos Opacos}: Solo ocurre \hl{conducción}. Sin embargo, en sus superficies expuestas a un fluido o a otras superficies, puede haber \hl{convección y/o radiación}.
        
        \item \textbf{Sólidos Semitransparentes}: Pueden comprender \hl{conducción y radiación}.
        
        \item \textbf{Fluidos Estáticos (sin movimiento masivo)}: La transferencia de calor es por \hl{conducción y, posiblemente, por radiación}.
        \begin{itemize}
            \item Los \hl{gases} son prácticamente transparentes a la radiación (actúan como el vacío).
            \item Los \hl{líquidos} suelen ser fuertes absorbentes de radiación.
        \end{itemize}
        
        \item \textbf{Fluidos en Movimiento}: La transferencia de calor es por \hl{convección y radiación}. La convección puede verse como una combinación de conducción y movimiento de fluido.
        
        \item \textbf{Vacío}: La transferencia de calor solo ocurre por \hl{radiación}, ya que la conducción y la convección requieren de un medio material.
    \end{itemize}

    \subsection{Aplicaciones en Ingeniería Automotriz}

    La transferencia de calor es omnipresente en la ingeniería automotriz. Algunos ejemplos clave incluyen:

    \subsubsection{Sistema de Enfriamiento del Motor}
    \begin{itemize}
        \item \textbf{Conducción}: Transferencia desde los gases de combustión hacia las paredes del cilindro
        \item \textbf{Convección forzada}: Circulación del líquido refrigerante
        \item \textbf{Radiación}: Disipación de calor en el radiador
    \end{itemize}

    \subsubsection{Sistema de Frenos}
    \begin{itemize}
        \item \textbf{Conducción}: Transferencia de calor generado por fricción
        \item \textbf{Convección}: Enfriamiento por aire a través de discos ventilados
        \item \textbf{Radiación}: Pérdida de calor a altas temperaturas
    \end{itemize}

    \subsubsection{Sistema de Escape}
    \begin{itemize}
        \item \textbf{Conducción}: A través de las paredes del múltiple
        \item \textbf{Convección}: Hacia el aire circundante
        \item \textbf{Radiación}: Especialmente significativa a altas temperaturas
    \end{itemize}

    \subsubsection{Aplicaciones Generales}
    \begin{itemize}
        \item Diseño de \hl{sistemas de calefacción y aire acondicionado}
        \item \hl{Radiadores} de automóviles y \hl{colectores solares}
        \item Determinación del espesor óptimo de \hl{aislamiento} en componentes automotrices
        \item Gestión térmica de \hl{componentes electrónicos} del vehículo
    \end{itemize}

    \section{Ejercicios de Reforzamiento}

    \subsection{Ejercicio 1: Transferencia de Calor por Conducción}

    \textbf{Problema:} Una ventana de vidrio tiene \textbf{1.5 m de alto, 0.8 m de ancho y un espesor de 0.5 cm}. Si las temperaturas de sus superficies interior y exterior son de \textbf{20°C y 5°C}, respectivamente, y la conductividad térmica del vidrio es \textbf{0.78 W/m·°C}, determine la razón de la pérdida de calor a través de esta ventana en estado estacionario.

    \textbf{Solución:}

    Primero, calculamos el área de transferencia de calor (A) y convertimos el espesor a metros:

    \begin{itemize}
        \item Área de la ventana (A) = alto × ancho = 1.5 m × 0.8 m = \textbf{1.2 m²}.
        \item Espesor del vidrio (L) = 0.5 cm = \textbf{0.005 m}.
    \end{itemize}

    Aplicamos la fórmula para la razón de conducción del calor a través de una capa plana: 

    \begin{equation}
    \dot{Q}_{cond} = \frac{kA (T_1 - T_2)}{L}
    \end{equation}

    Sustituyendo los valores:
    \begin{align}
    \dot{Q}_{cond} &= \frac{(0.78 \text{ W/m·°C}) \times (1.2 \text{ m²}) \times (20°C - 5°C)}{0.005 \text{ m}} \\
    &= \frac{0.78 \times 1.2 \times 15}{0.005} \text{ W} \\
    &= \frac{14.04}{0.005} \text{ W} \\
    &= \textbf{2808 W}
    \end{align}

    \textbf{Respuesta:} La ventana está perdiendo \textbf{2808 W} de calor.

    \subsection{Ejercicio 2: Transferencia de Calor por Convección}

    \textbf{Problema:} Una persona está de pie en una habitación donde la temperatura del aire es de \textbf{22°C}. Si el área superficial expuesta de la persona es de \textbf{1.7 m²} y su temperatura superficial promedio es de \textbf{30°C}, determine la razón de transferencia de calor por convección de la persona al aire. Considere un coeficiente de transferencia de calor por convección de \textbf{6 W/m²·°C}.

    \textbf{Solución:}

    Aplicamos la Ley de Newton del Enfriamiento:

    \begin{equation}
    \dot{Q}_{conv} = h \times A_s \times (T_s - T_\infty)
    \end{equation}

    Sustituyendo los valores:
    \begin{align}
    \dot{Q}_{conv} &= (6 \text{ W/m²·°C}) \times (1.7 \text{ m²}) \times (30°C - 22°C) \\
    &= 6 \times 1.7 \times 8 \text{ W} \\
    &= \textbf{81.6 W}
    \end{align}

    \textbf{Respuesta:} La persona está perdiendo \textbf{81.6 W} de calor al aire por convección.

    \subsection{Ejercicio 3: Transferencia de Calor por Radiación}

    \textbf{Problema:} Un pequeño objeto esférico negro (con una emisividad de $\varepsilon = 1$, un cuerpo negro ideal) con un área superficial de $0.1 m^2$ está a una temperatura de \textbf{150°C}. Este objeto está completamente rodeado por superficies que se encuentran a una temperatura uniforme de \textbf{25°C}. Determine la razón neta de transferencia de calor por radiación del objeto a sus alrededores.

    \textbf{Solución:}

    Para calcular la razón neta de transferencia de calor por radiación, es crucial usar \textbf{temperaturas absolutas (Kelvin)}.

    \begin{itemize}
        \item Convertimos $T_s$ a Kelvin: $T_s$ (K) = 150°C + 273.15 = 423.15 K.
        \item Convertimos $T_{alred}$ a Kelvin: $T_{alred}$ (K) = 25°C + 273.15 = 298.15 K.
    \end{itemize}

    Aplicamos la fórmula para la razón neta de transferencia de calor por radiación:

    \begin{equation}
    \dot{Q}_{rad} = \varepsilon \times A_s \times \sigma \times (T_s^4 - T_{alred}^4)
    \end{equation}

    Sustituyendo los valores:
    \begin{align}
    \dot{Q}_{rad} &= 1 \times (0.1 \,  m^2) \times (5.67 \times 10^{-8}  \, W/m^2 \cdot K^4) \times [(423.15 \text{ K})^4 - (298.15 \text{ K})^4] \\
    &= 0.1 \times 5.67 \times 10^{-8} \times (3,200,770,630.06 - 789,230,670.06) \text{ W} \\
    &= 0.1 \times 5.67 \times 10^{-8} \times (2,411,539,960) \text{ W} \\
    &\approx \textbf{136.6 W}
    \end{align}

    \textbf{Respuesta:} El objeto está perdiendo aproximadamente \textbf{136.6 W} de calor por radiación a sus alrededores.

    \subsection{Ejercicio 4: Aplicación Automotriz - Sistema de Enfriamiento}

    \textbf{Problema:} El bloque de un motor de automóvil está fabricado de hierro fundido con una \textbf{conductividad térmica k = 52 W/m·K}. La pared del cilindro tiene un espesor de \textbf{8 mm}. Si la temperatura de la superficie interna (en contacto con los gases de combustión) es de \textbf{400°C} y la temperatura de la superficie externa (en contacto con el refrigerante) debe mantenerse a \textbf{90°C} para evitar la ebullición, determine la \textbf{razón de transferencia de calor por unidad de área} a través de la pared del cilindro.

    \textbf{Solución:}

    Para transferencia de calor por conducción unidimensional en estado estacionario:

    \begin{equation}
    \dot{q} = \frac{\dot{Q}}{A} = \frac{k(T_1 - T_2)}{L}
    \end{equation}

    Sustituyendo los valores:
    \begin{align}
    \dot{q} &= \frac{(52 \text{ W/m·K})(400 - 90)°C}{0.008 \text{ m}} \\
    &= \frac{(52)(310)}{0.008} \\
    &= \textbf{2,015,000 \text{ W/m²} = 2.015 \text{ MW/m²}}
    \end{align}

    \textbf{Respuesta:} La razón de transferencia de calor por unidad de área es \textbf{2.015 MW/m²}, lo que demuestra la intensa transferencia de calor que debe gestionarse en un motor de combustión interna.

    \subsection{Ejercicio 5: Aplicación Automotriz - Sistema de Frenos}

    \textbf{Problema:} Durante una frenada intensa, un disco de freno alcanza una temperatura superficial de \textbf{350°C}. Si el disco tiene un área expuesta de $0.12 m^2$ y una \textbf{emisividad $\varepsilon = 0.7$}, calcule la \textbf{razón de pérdida de calor por radiación} cuando la temperatura ambiente es de \textbf{30°C}.

    \textbf{Solución:}

    Convertimos a temperaturas absolutas:
    \begin{itemize}
        \item $T_s = 350°C = 623.15 K$
        \item $T_{amb} = 30°C = 303.15 K$
    \end{itemize}

    Aplicando la ley de Stefan-Boltzmann:

    \begin{equation}
    \dot{Q}_{rad} = \varepsilon\sigma A_s(T_s^4 - T_{amb}^4)
    \end{equation}

    Sustituyendo los valores:
    \begin{align}
    \dot{Q}_{rad} &= (0.7)(5.67 \times 10^{-8})(0.12)[(623.15)^4 - (303.15)^4] \\
    &= (4.76 \times 10^{-9})[1.509 \times 10^{11} - 8.44 \times 10^9] \\
    &= (4.76 \times 10^{-9})[1.425 \times 10^{11}] \\
    &= \textbf{678 W}
    \end{align}

    \textbf{Respuesta:} El disco de freno pierde \textbf{678 W} por radiación, un mecanismo importante de enfriamiento a altas temperaturas.

    \section{Conclusión}

    La \hl{transferencia de calor} constituye una disciplina fundamental para el ingeniero automotriz, proporcionando las herramientas teóricas y prácticas necesarias para el diseño, análisis y optimización de \hl{sistemas térmicos} en vehículos modernos.

    A través del estudio detallado de los \hl{conceptos fundamentales} presentados en esta lectura, el estudiante ha adquirido comprensión sobre:

    \begin{enumerate}
        \item \textbf{La distinción crucial} entre termodinámica y transferencia de calor, donde la primera se enfoca en las \hl{cantidades} de energía intercambiada entre estados de equilibrio, mientras que la segunda se centra en las \hl{velocidades} de estos intercambios energéticos.

        \item \textbf{Los tres mecanismos básicos} de transferencia de calor:
        \begin{itemize}
            \item \textbf{Conducción}: Transferencia a través de materiales sólidos, gobernada por la \hl{Ley de Fourier}
            \item \textbf{Convección}: Transferencia entre superficies sólidas y fluidos en movimiento, descrita por la \hl{Ley de Newton del Enfriamiento}
            \item \textbf{Radiación}: Transferencia mediante ondas electromagnéticas, regida por la \hl{Ley de Stefan-Boltzmann}
        \end{itemize}

        \item \textbf{Las aplicaciones prácticas} en sistemas automotrices, desde la gestión térmica del motor hasta el enfriamiento de frenos, demostrando la relevancia directa de estos principios en la ingeniería de vehículos.

        \item \textbf{La importancia del balance de energía} basado en la Primera Ley de la Termodinámica para el análisis sistemático de sistemas térmicos complejos.
    \end{enumerate}

    El dominio de estos conceptos prepara al futuro ingeniero para enfrentar los desafíos de la \hl{eficiencia energética}, la \hl{gestión térmica} y el \hl{diseño sostenible} en la industria automotriz moderna. La capacidad de \hl{cuantificar} y \hl{predecir} los fenómenos de transferencia de calor es esencial para el desarrollo de tecnologías más \hl{eficientes}, \hl{seguras} y \hl{ambientalmente responsables}.

    En el contexto actual de \hl{transición hacia la movilidad eléctrica} y \hl{tecnologías limpias}, estos fundamentos cobran mayor relevancia para el diseño de \hl{sistemas de propulsión alternativos}, \hl{baterías}, \hl{sistemas de climatización eficientes} y \hl{componentes electrónicos} que definirán el futuro del transporte.

    \section{Bibliografía}

    Bergman, T. L., Lavine, A. S., Incropera, F. P., \& DeWitt, D. P. (2017). \emph{Fundamentals of heat and mass transfer} (8th ed.). John Wiley \& Sons.

    Çengel, Y. A. (2020). \emph{Heat and mass transfer: Fundamentals and applications} (6th ed.). McGraw-Hill Education.

    Çengel, Y. A., \& Boles, M. A. (2019). \emph{Termodinámica} (8ª ed.). McGraw-Hill Interamericana.

    Heywood, J. B. (2018). \emph{Internal combustion engine fundamentals} (2nd ed.). McGraw-Hill Education.

    Holman, J. P. (2018). \emph{Heat transfer} (10th ed.). McGraw-Hill Education.

    Lienhard IV, J. H., \& Lienhard V, J. H. (2019). \emph{A heat transfer textbook} (5th ed.). Phlogiston Press.

    Moran, M. J., Shapiro, H. N., Boettner, D. D., \& Bailey, M. B. (2018). \emph{Fundamentals of engineering thermodynamics} (9th ed.). John Wiley \& Sons.

    Payri, F., \& Desantes, J. M. (Coords.). (2011). \emph{Motores de combustión interna alternativos}. Editorial de la Universidad Politécnica de Valencia.

    Stone, R. (2012). \emph{Introduction to internal combustion engines} (4th ed.). SAE International.

    White, F. M. (2016). \emph{Heat and mass transfer} (2nd ed.). Addison-Wesley.

\end{document}