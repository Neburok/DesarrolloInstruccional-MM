\documentclass[a4paper,landscape]{article}
\usepackage[utf8]{inputenc}
\usepackage[spanish]{babel}
\usepackage{tikz}
\usepackage{xcolor}
\usepackage{geometry}
\usepackage{amsmath}
\usepackage{helvet}
\usepackage{graphicx}

% Configuración de página
\geometry{margin=1cm}
\renewcommand{\familydefault}{\sfdefault}
\pagestyle{empty}

% Colores institucionales UTP
\definecolor{UTPGreen}{HTML}{00833D}
\definecolor{UTPLightGreen}{HTML}{4CAF50}
\definecolor{UTPDarkGreen}{HTML}{2E7D32}
\definecolor{UTPGray}{HTML}{666666}
\definecolor{UTPLightGray}{HTML}{E0E0E0}
\definecolor{UTPBlue}{HTML}{1976D2}
\definecolor{UTPOrange}{HTML}{FF9800}
\definecolor{UTPRed}{HTML}{D32F2F}

\usetikzlibrary{shapes.geometric, arrows, positioning, decorations.pathreplacing}

\begin{document}

\begin{tikzpicture}[remember picture, overlay]
    % Fondo principal
    \fill[UTPLightGray!20] (current page.south west) rectangle (current page.north east);
    
    % Header institucional
    \fill[UTPGreen] (current page.north west) rectangle ([yshift=-2cm]current page.north east);
    
    % Título principal
    \node[anchor=north, text=white, font=\Huge\bfseries] at ([yshift=-1cm]current page.north) 
        {PROCESOS TERMODINÁMICOS FUNDAMENTALES};
    
    % Subtítulo
    \node[anchor=north, text=white, font=\Large] at ([yshift=-1.5cm]current page.north) 
        {Ingeniería Automotriz - Termodinámica};
    
    % Logo UTP (placeholder)
    \node[anchor=north west, text=white, font=\large\bfseries] at ([xshift=1cm, yshift=-0.5cm]current page.north west) 
        {UTP};
\end{tikzpicture}

% Contenido principal
\vspace{3cm}

\begin{tikzpicture}[
    process/.style={rectangle, rounded corners=10pt, minimum width=6cm, minimum height=8cm, 
                   text centered, draw=black, line width=2pt, fill=#1},
    equation/.style={rectangle, rounded corners=5pt, minimum width=5cm, minimum height=1cm,
                    text centered, draw=UTPGray, fill=white, font=\footnotesize},
    arrow/.style={thick, ->, >=stealth}
]

% Proceso Isotérmico
\node[process=UTPBlue!20] (isotermico) at (0,0) {
    \begin{minipage}{5.5cm}
        \centering
        {\color{UTPBlue}\Huge\bfseries ISOTÉRMICO}\\[0.3cm]
        {\color{UTPDarkGreen}\Large\bfseries T = constante}\\[0.5cm]
        
        \includegraphics[width=3cm]{diagrama-isotermico.png}\\[0.3cm]
        
        {\color{UTPGray}\small\textbf{Características:}}\\
        {\color{black}\small • Temperatura constante}\\
        {\color{black}\small • Intercambio de calor con}\\
        {\color{black}\small \phantom{•} el ambiente}\\
        {\color{black}\small • Proceso muy lento}\\[0.3cm]
        
        {\color{UTPGray}\small\textbf{Aplicaciones:}}\\
        {\color{black}\small • Expansión de gases}\\
        {\color{black}\small • Compresores lentos}\\
        {\color{black}\small • Procesos reversibles}\\[0.3cm]
        
        \begin{tikzpicture}
            \node[equation] {\color{UTPBlue}$PV = \text{constante}$};
        \end{tikzpicture}
    \end{minipage}
};

% Proceso Isocórico
\node[process=UTPOrange!20] (isocorico) at (7,0) {
    \begin{minipage}{5.5cm}
        \centering
        {\color{UTPOrange}\Huge\bfseries ISOCÓRICO}\\[0.3cm]
        {\color{UTPDarkGreen}\Large\bfseries V = constante}\\[0.5cm]
        
        \includegraphics[width=3cm]{diagrama-isocorico.png}\\[0.3cm]
        
        {\color{UTPGray}\small\textbf{Características:}}\\
        {\color{black}\small • Volumen constante}\\
        {\color{black}\small • Sin trabajo de frontera}\\
        {\color{black}\small • Calentamiento/enfriamiento}\\
        {\color{black}\small \phantom{•} a volumen fijo}\\[0.3cm]
        
        {\color{UTPGray}\small\textbf{Aplicaciones:}}\\
        {\color{black}\small • Combustión en motores}\\
        {\color{black}\small • Calentamiento en}\\
        {\color{black}\small \phantom{•} recipientes rígidos}\\[0.3cm]
        
        \begin{tikzpicture}
            \node[equation] {\color{UTPOrange}$\frac{P}{T} = \text{constante}$};
        \end{tikzpicture}
    \end{minipage}
};

% Proceso Adiabático
\node[process=UTPRed!20] (adiabatico) at (14,0) {
    \begin{minipage}{5.5cm}
        \centering
        {\color{UTPRed}\Huge\bfseries ADIABÁTICO}\\[0.3cm]
        {\color{UTPDarkGreen}\Large\bfseries Q = 0}\\[0.5cm]
        
        \includegraphics[width=3cm]{diagrama-adiabatico.png}\\[0.3cm]
        
        {\color{UTPGray}\small\textbf{Características:}}\\
        {\color{black}\small • Sin intercambio de calor}\\
        {\color{black}\small • Proceso muy rápido}\\
        {\color{black}\small • Sistema aislado}\\
        {\color{black}\small • Cambio de temperatura}\\[0.3cm]
        
        {\color{UTPGray}\small\textbf{Aplicaciones:}}\\
        {\color{black}\small • Compresión rápida}\\
        {\color{black}\small • Expansión en turbinas}\\
        {\color{black}\small • Motores de combustión}\\[0.3cm]
        
        \begin{tikzpicture}
            \node[equation] {\color{UTPRed}$PV^{\gamma} = \text{constante}$};
        \end{tikzpicture}
    \end{minipage}
};

% Proceso Isobárico
\node[process=UTPLightGreen!20] (isobarico) at (21,0) {
    \begin{minipage}{5.5cm}
        \centering
        {\color{UTPLightGreen}\Huge\bfseries ISOBÁRICO}\\[0.3cm]
        {\color{UTPDarkGreen}\Large\bfseries P = constante}\\[0.5cm]
        
        \includegraphics[width=3cm]{diagrama-isobarico.png}\\[0.3cm]
        
        {\color{UTPGray}\small\textbf{Características:}}\\
        {\color{black}\small • Presión constante}\\
        {\color{black}\small • Expansión/compresión}\\
        {\color{black}\small \phantom{•} controlada}\\
        {\color{black}\small • Intercambio de calor}\\[0.3cm]
        
        {\color{UTPGray}\small\textbf{Aplicaciones:}}\\
        {\color{black}\small • Combustión en diesel}\\
        {\color{black}\small • Calentamiento con}\\
        {\color{black}\small \phantom{•} pistón móvil}\\[0.3cm]
        
        \begin{tikzpicture}
            \node[equation] {\color{UTPLightGreen}$\frac{V}{T} = \text{constante}$};
        \end{tikzpicture}
    \end{minipage}
};

\end{tikzpicture}

% Sección de comparación
\vspace{2cm}

\begin{tikzpicture}[remember picture, overlay]
    % Franja de comparación
    \fill[UTPGray!20] ([yshift=-13cm]current page.north west) rectangle ([yshift=-18cm]current page.north east);
    
    % Título de comparación
    \node[anchor=north, text=UTPDarkGreen, font=\LARGE\bfseries] at ([yshift=-13.5cm]current page.north) 
        {TABLA COMPARATIVA DE PROCESOS};
\end{tikzpicture}

\vspace{3cm}

% Tabla comparativa
\begin{center}
\begin{tikzpicture}
\node[anchor=center] at (0,0) {
    \begin{tabular}{|c|c|c|c|c|}
    \hline
    \rowcolor{UTPGreen!80}
    \textcolor{white}{\textbf{Proceso}} & 
    \textcolor{white}{\textbf{Constante}} & 
    \textcolor{white}{\textbf{Ecuación}} & 
    \textcolor{white}{\textbf{Trabajo}} & 
    \textcolor{white}{\textbf{Calor}} \\
    \hline
    \rowcolor{UTPBlue!20}
    \textbf{Isotérmico} & T = cte & $PV = \text{cte}$ & $W = nRT\ln\frac{V_f}{V_i}$ & $Q = W$ \\
    \hline
    \rowcolor{UTPOrange!20}
    \textbf{Isocórico} & V = cte & $\frac{P}{T} = \text{cte}$ & $W = 0$ & $Q = nC_V\Delta T$ \\
    \hline
    \rowcolor{UTPRed!20}
    \textbf{Adiabático} & Q = 0 & $PV^{\gamma} = \text{cte}$ & $W = \frac{P_iV_i - P_fV_f}{\gamma-1}$ & $Q = 0$ \\
    \hline
    \rowcolor{UTPLightGreen!20}
    \textbf{Isobárico} & P = cte & $\frac{V}{T} = \text{cte}$ & $W = P\Delta V$ & $Q = nC_P\Delta T$ \\
    \hline
    \end{tabular}
};
\end{tikzpicture}
\end{center}

% Footer institucional
\begin{tikzpicture}[remember picture, overlay]
    \fill[UTPGreen] ([yshift=1cm]current page.south west) rectangle (current page.south east);
    
    \node[anchor=south, text=white, font=\large\bfseries] at ([yshift=0.5cm]current page.south) 
        {Universidad Tecnológica de Puebla - Ingeniería Automotriz};
\end{tikzpicture}

\end{document}