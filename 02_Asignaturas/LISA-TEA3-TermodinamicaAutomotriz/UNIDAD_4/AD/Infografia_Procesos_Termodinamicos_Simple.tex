\documentclass[a3paper,landscape]{article}
\usepackage[utf8]{inputenc}
\usepackage{tikz}
\usepackage[spanish,es-nodecimaldot]{babel}

% Fix for babel and TikZ compatibility
\usetikzlibrary{babel}
\usepackage{xcolor}
\usepackage{geometry}
\usepackage{amsmath}
\usepackage{helvet}
\usepackage{fontawesome5}

% Configuración de página
\geometry{margin=0.5cm}
\renewcommand{\familydefault}{\sfdefault}
\pagestyle{empty}

% Colores institucionales UTP
\definecolor{UTPGreen}{HTML}{00833D}
\definecolor{UTPLightGreen}{HTML}{4CAF50}
\definecolor{UTPDarkGreen}{HTML}{2E7D32}
\definecolor{UTPGray}{HTML}{666666}
\definecolor{UTPLightGray}{HTML}{F5F5F5}
\definecolor{UTPBlue}{HTML}{1976D2}
\definecolor{UTPOrange}{HTML}{FF9800}
\definecolor{UTPRed}{HTML}{D32F2F}
\definecolor{UTPPurple}{HTML}{7B1FA2}

\usetikzlibrary{shapes.geometric, arrows, positioning, decorations.pathreplacing, calc, shadows}

\begin{document}

% Header
\begin{tikzpicture}[remember picture, overlay]
    \fill[UTPGreen] (current page.north west) rectangle ([yshift=-3cm]current page.north east);
    
    % Título principal
    \node[text=white, font=\fontsize{36}{40}\selectfont\bfseries] at ([yshift=-1.5cm]current page.north) 
        {PROCESOS TERMODINÁMICOS FUNDAMENTALES};
    
    % Subtítulo
    \node[text=white!80, font=\Large] at ([yshift=-2.3cm]current page.north) 
        {Termodinámica Automotriz - Universidad Tecnológica de Puebla};
\end{tikzpicture}

\vspace{4cm}

% Contenido principal
\begin{center}
\begin{tikzpicture}[
    process/.style={rectangle, rounded corners=15pt, minimum width=8.5cm, minimum height=12cm, 
                   text centered, draw=black!30, line width=3pt, fill=#1, drop shadow},
    header/.style={rectangle, rounded corners=10pt, minimum width=7cm, minimum height=1.5cm,
                  text centered, fill=#1, text=white, font=\Large\bfseries},
    equation/.style={rectangle, rounded corners=8pt, minimum width=6cm, minimum height=1.2cm,
                    text centered, fill=white, draw=#1, line width=2pt, font=\large},
    feature/.style={font=\normalsize, text width=6cm, align=left}
]

% ISOTÉRMICO
\node[process=UTPBlue!15] (isotermico) at (-12,0) {};
\node[header=UTPBlue] at (-12,5) {ISOTÉRMICO};
\node[text=UTPDarkGreen, font=\LARGE\bfseries] at (-12,4) {T = constante};

% Diagrama P-V isotérmico
\begin{scope}[shift={(-12,2)}]
    \draw[thick, ->] (-1.5,0) -- (1.5,0) node[right] {\small V};
    \draw[thick, ->] (0,-1) -- (0,1.5) node[above] {\small P};
    \draw[UTPBlue, very thick, domain=-1.2:1.2, samples=100] plot (\x, {1/(\x+2)});
    \node[UTPBlue, font=\small\bfseries] at (0.8,1.2) {$PV = cte$};
\end{scope}

\node[feature] at (-12,0.2) {
    \textcolor{UTPGray}{\textbf{Características:}}\\
    • Temperatura constante\\
    • Intercambio de calor\\
    • Proceso muy lento\\
    • Equilibrio térmico
};

\node[feature] at (-12,-1.5) {
    \textcolor{UTPGray}{\textbf{Aplicaciones:}}\\
    • Expansión controlada\\
    • Compresores lentos\\
    • Procesos reversibles
};

\node[equation=UTPBlue] at (-12,-3.5) {$W = nRT\ln\frac{V_f}{V_i}$};
\node[equation=UTPBlue] at (-12,-4.8) {$Q = W$};

% ISOCÓRICO
\node[process=UTPOrange!15] (isocorico) at (-4,0) {};
\node[header=UTPOrange] at (-4,5) {ISOCÓRICO};
\node[text=UTPDarkGreen, font=\LARGE\bfseries] at (-4,4) {V = constante};

% Diagrama P-V isocórico
\begin{scope}[shift={(-4,2)}]
    \draw[thick, ->] (-1.5,0) -- (1.5,0) node[right] {\small V};
    \draw[thick, ->] (0,-1) -- (0,1.5) node[above] {\small P};
    \draw[UTPOrange, very thick] (0.5,-0.8) -- (0.5,1.3);
    \node[UTPOrange, font=\small\bfseries] at (0.8,1.2) {$V = cte$};
\end{scope}

\node[feature] at (-4,0.2) {
    \textcolor{UTPGray}{\textbf{Características:}}\\
    • Volumen constante\\
    • Sin trabajo de frontera\\
    • Solo intercambio de calor\\
    • Recipiente rígido
};

\node[feature] at (-4,-1.5) {
    \textcolor{UTPGray}{\textbf{Aplicaciones:}}\\
    • Combustión en motores\\
    • Calentamiento en tanques\\
    • Procesos a V constante
};

\node[equation=UTPOrange] at (-4,-3.5) {$W = 0$};
\node[equation=UTPOrange] at (-4,-4.8) {$Q = nC_V\Delta T$};

% ADIABÁTICO
\node[process=UTPRed!15] (adiabatico) at (4,0) {};
\node[header=UTPRed] at (4,5) {ADIABÁTICO};
\node[text=UTPDarkGreen, font=\LARGE\bfseries] at (4,4) {Q = 0};

% Diagrama P-V adiabático
\begin{scope}[shift={(4,2)}]
    \draw[thick, ->] (-1.5,0) -- (1.5,0) node[right] {\small V};
    \draw[thick, ->] (0,-1) -- (0,1.5) node[above] {\small P};
    \draw[UTPRed, very thick, domain=-1:1.2, samples=100] plot (\x, {1.2/(\x+2)^1.4});
    \node[UTPRed, font=\small\bfseries] at (0.8,1.2) {$PV^{\gamma} = cte$};
\end{scope}

\node[feature] at (4,0.2) {
    \textcolor{UTPGray}{\textbf{Características:}}\\
    • Sin intercambio de calor\\
    • Proceso muy rápido\\
    • Sistema aislado\\
    • Cambio de temperatura
};

\node[feature] at (4,-1.5) {
    \textcolor{UTPGray}{\textbf{Aplicaciones:}}\\
    • Compresión rápida\\
    • Expansión en turbinas\\
    • Motores de combustión
};

\node[equation=UTPRed] at (4,-3.5) {$W = \frac{P_iV_i - P_fV_f}{\gamma-1}$};
\node[equation=UTPRed] at (4,-4.8) {$Q = 0$};

% ISOBÁRICO
\node[process=UTPLightGreen!15] (isobarico) at (12,0) {};
\node[header=UTPLightGreen] at (12,5) {ISOBÁRICO};
\node[text=UTPDarkGreen, font=\LARGE\bfseries] at (12,4) {P = constante};

% Diagrama P-V isobárico
\begin{scope}[shift={(12,2)}]
    \draw[thick, ->] (-1.5,0) -- (1.5,0) node[right] {\small V};
    \draw[thick, ->] (0,-1) -- (0,1.5) node[above] {\small P};
    \draw[UTPLightGreen, very thick] (-1,0.6) -- (1.2,0.6);
    \node[UTPLightGreen, font=\small\bfseries] at (0.8,1.2) {$P = cte$};
\end{scope}

\node[feature] at (12,0.2) {
    \textcolor{UTPGray}{\textbf{Características:}}\\
    • Presión constante\\
    • Expansión controlada\\
    • Intercambio de calor\\
    • Trabajo proporcional a $\Delta V$
};

\node[feature] at (12,-1.5) {
    \textcolor{UTPGray}{\textbf{Aplicaciones:}}\\
    • Combustión en Diesel\\
    • Calentamiento a P constante\\
    • Procesos con pistón libre
};

\node[equation=UTPLightGreen] at (12,-3.5) {$W = P\Delta V$};
\node[equation=UTPLightGreen] at (12,-4.8) {$Q = nC_P\Delta T$};

\end{tikzpicture}
\end{center}

% Sección de fórmulas importantes
\vspace{1cm}
\begin{tikzpicture}[remember picture, overlay]
    \fill[UTPGray!10] ([yshift=-20cm]current page.north west) rectangle ([yshift=-26cm]current page.north east);
    
    \node[text=UTPDarkGreen, font=\LARGE\bfseries] at ([yshift=-21cm]current page.north) 
        {ECUACIONES FUNDAMENTALES};
\end{tikzpicture}

\vspace{2cm}

\begin{center}
\begin{tikzpicture}[
    formula/.style={rectangle, rounded corners=10pt, minimum width=12cm, minimum height=2cm,
                   text centered, fill=#1, text=white, font=\Large\bfseries}
]

\node[formula=UTPBlue] at (-8,0) {
    Primera Ley: $\Delta U = Q - W$
};

\node[formula=UTPOrange] at (0,0) {
    Gas Ideal: $PV = nRT$
};

\node[formula=UTPRed] at (8,0) {
    Adiabática: $TV^{\gamma-1} = cte$
};

\node[formula=UTPDarkGreen] at (-4,-2.5) {
    Trabajo: $W = \int P \, dV$
};

\node[formula=UTPPurple] at (4,-2.5) {
    Calor: $Q = nC\Delta T$
};

\end{tikzpicture}
\end{center}

% Footer
\begin{tikzpicture}[remember picture, overlay]
    \fill[UTPGreen] ([yshift=1cm]current page.south west) rectangle (current page.south east);
    
    \node[text=white, font=\Large\bfseries] at ([yshift=0.5cm]current page.south) 
        {Universidad Tecnológica de Puebla - Ingeniería Automotriz};
\end{tikzpicture}

\end{document}