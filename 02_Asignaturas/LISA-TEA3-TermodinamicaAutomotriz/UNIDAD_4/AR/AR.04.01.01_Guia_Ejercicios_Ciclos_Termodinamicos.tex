\documentclass{article}

% --- Cargar la plantilla de estilo UTP ---
\usepackage{utp-doc}
\usepackage{tikz}
\usepackage{amsmath}

\begin{document}

% --- Título de la Guía ---
\practicatitle{Guía de Ejercicios Resueltos: Ciclos Termodinámicos}

% --- Metadatos de la Actividad ---
\textbf{Asignatura:} Termodinámica Automotriz \\
\textbf{Unidad 4:} Procesos Termodinámicos y de Transferencia de Calor

\vspace{5mm}
\hrule
\vspace{5mm}

\section*{Introducción}

Esta guía de ejercicios está diseñada para reforzar su comprensión de los ciclos termodinámicos fundamentales: Carnot, Otto y Diesel. La resolución de estos problemas le permitirá aplicar los conceptos teóricos aprendidos en la lectura y desarrollar sus habilidades de cálculo y análisis. Cada ejercicio incluye una solución detallada paso a paso para facilitar su aprendizaje.

\section*{Ejercicio 1: Eficiencia de un Ciclo de Carnot}

Un motor térmico opera siguiendo un ciclo de Carnot entre una fuente de calor a $900 \, K$ y un sumidero de calor a $300 \, K$.

a) Calcule la eficiencia térmica de este motor.

b) Si el motor absorbe $1500 \, kJ$ de calor de la fuente caliente por ciclo, ¿cuánto trabajo neto produce y cuánto calor rechaza al sumidero?

\subsection*{Solución Detallada}

a) \textbf{Eficiencia Térmica del Ciclo de Carnot:}

La eficiencia térmica de un ciclo de Carnot se calcula utilizando las temperaturas absolutas de la fuente caliente ($T_H$) y el sumidero frío ($T_L$):

$$ \eta_{th,Carnot} = 1 - \frac{T_L}{T_H} $$

Sustituyendo los valores dados:
$T_H = 900 \, K$
$T_L = 300 \, K$

$$ \eta_{th,Carnot} = 1 - \frac{300 \, K}{900 \, K} = 1 - \frac{1}{3} = \frac{2}{3} $$

$$ \eta_{th,Carnot} \approx 0.6667 \quad \text{o} \quad 66.67\% $$

La eficiencia térmica de este motor de Carnot es del $66.67\%$.

b) \textbf{Trabajo Neto Producido y Calor Rechazado:}

Sabemos que la eficiencia térmica también se define como la relación entre el trabajo neto producido ($W_{neto}$) y el calor absorbido de la fuente caliente ($Q_H$):

$$ \eta_{th} = \frac{W_{neto}}{Q_H} $$

Podemos despejar el trabajo neto:

$$ W_{neto} = \eta_{th} \times Q_H $$

Sustituyendo los valores:
$Q_H = 1500 \, kJ$

$$ W_{neto} = 0.6667 \times 1500 \, kJ = 1000.05 \, kJ $$

El trabajo neto producido por el motor es de aproximadamente $1000 \, kJ$.

Para calcular el calor rechazado al sumidero ($Q_L$), utilizamos el balance de energía para un ciclo:

$$ W_{neto} = Q_H - Q_L $$

Despejando $Q_L$:

$$ Q_L = Q_H - W_{neto} $$

Sustituyendo los valores:

$$ Q_L = 1500 \, kJ - 1000 \, kJ = 500 \, kJ $$

El calor rechazado al sumidero es de $500 \, kJ$.

\end{document}
