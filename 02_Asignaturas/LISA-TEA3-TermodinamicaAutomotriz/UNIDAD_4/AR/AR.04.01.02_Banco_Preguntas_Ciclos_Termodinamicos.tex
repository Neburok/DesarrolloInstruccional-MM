\documentclass{article}

% --- Cargar la plantilla de estilo UTP ---
\usepackage{utp-doc}
\usepackage{tikz}
\usepackage{amsmath}
\usepackage{enumitem}

\begin{document}

% --- Título del Banco de Preguntas ---
\practicatitle{Banco de Preguntas de Opción Múltiple: Ciclos Termodinámicos}

% --- Metadatos de la Actividad ---
\textbf{Asignatura:} Termodinámica Automotriz \\
\textbf{Unidad 4:} Procesos Termodinámicos y de Transferencia de Calor

\vspace{5mm}
\hrule
\vspace{5mm}

\section*{Introducción}

Este banco de preguntas le permitirá evaluar su comprensión de los conceptos clave relacionados con los ciclos termodinámicos de Carnot, Otto y Diesel. Seleccione la opción que considere correcta para cada pregunta.

\section*{Preguntas}

\begin{enumerate}[label=\arabic*.]

    \item ¿Cuál es el propósito principal de un ciclo termodinámico?
    \begin{enumerate}[label=\alph*)]
        \item Almacenar energía térmica.
        \item Convertir trabajo en calor.
        \item \textbf{Convertir calor en trabajo mecánico de forma continua.}
        \item Reducir la temperatura de un sistema.
    \end{enumerate}

    \item En un diagrama Presión-Volumen (P-V), ¿qué representa el área encerrada por la curva de un ciclo termodinámico?
    \begin{enumerate}[label=\alph*)]
        \item El calor total transferido.
        \item \textbf{El trabajo neto realizado por o sobre el sistema.}
        \item La eficiencia térmica del ciclo.
        \item La temperatura máxima alcanzada.
    \end{enumerate}

    \item ¿Qué característica define a un ciclo termodinámico reversible?
    \begin{enumerate}[label=\alph*)]
        \item Que todos sus procesos son isotérmicos.
        \item \textbf{Que puede invertirse sin dejar rastro en el entorno.}
        \item Que opera a muy bajas temperaturas.
        \item Que no intercambia calor con el entorno.
    \end{enumerate}

    \item ¿Cuál de los siguientes ciclos termodinámicos establece el límite máximo de eficiencia para cualquier motor térmico que opere entre dos temperaturas dadas?
    \begin{enumerate}[label=\alph*)]
        \item Ciclo de Otto.
        \item Ciclo Diesel.
        \item \textbf{Ciclo de Carnot.}
        \item Ciclo Brayton.
    \end{enumerate}

    \item El Ciclo de Carnot consta de cuatro procesos. ¿Cuáles son?
    \begin{enumerate}[label=\alph*)]
        \item Dos isobáricos y dos isocóricos.
        \item Dos isotérmicos y dos isobáricos.
        \item \textbf{Dos isotérmicos y dos adiabáticos.}
        \item Dos isocóricos y dos adiabáticos.
    \end{enumerate}

    \item La eficiencia térmica de un ciclo de Carnot depende únicamente de:
    \begin{enumerate}[label=\alph*)]
        \item La presión y el volumen.
        \item El tipo de fluido de trabajo.
        \item \textbf{Las temperaturas absolutas de la fuente caliente y el sumidero frío.}
        \item La cantidad de calor absorbido.
    \end{enumerate}

    \item Si un motor de Carnot opera entre $T_H = 800 \, K$ y $T_L = 400 \, K$, ¿cuál es su eficiencia térmica?
    \begin{enumerate}[label=\alph*)]
        \item 25\%
        \item \textbf{50\%}
        \item 75\%
        \item 100\%
    \end{enumerate}

    \item ¿Qué tipo de motor de combustión interna es modelado idealmente por el Ciclo de Otto?
    \begin{enumerate}[label=\alph*)]
        \item Motores diésel.
        \item \textbf{Motores de encendido por chispa (gasolina).}
        \item Turbinas de gas.
        \item Motores de vapor.
    \end{enumerate}

    \item En el Ciclo de Otto, ¿cómo se modela la adición de calor?
    \begin{enumerate}[label=\alph*)]
        \item A presión constante.
        \item A temperatura constante.
        \item \textbf{A volumen constante.}
        \item Sin adición de calor.
    \end{enumerate}

    \item La eficiencia térmica ideal del Ciclo de Otto depende principalmente de:
    \begin{enumerate}[label=\alph*)]
        \item La relación de corte de inyección.
        \item \textbf{La relación de compresión.}
        \item La temperatura máxima del ciclo.
        \item La cantidad de combustible inyectado.
    \end{enumerate}

    \item ¿Qué tipo de motor de combustión interna es modelado idealmente por el Ciclo Diesel?
    \begin{enumerate}[label=\alph*)]
        \item Motores de encendido por chispa (gasolina).
        \item \textbf{Motores de encendido por compresión (diésel).}
        \item Motores eléctricos.
        \item Motores de reacción.
    \end{enumerate}

    \item La principal diferencia en la adición de calor entre el Ciclo de Otto y el Ciclo Diesel es que en el Diesel ocurre a:
    \begin{enumerate}[label=\alph*)]
        \item Volumen constante.
        \item Temperatura constante.
        \item \textbf{Presión constante.}
        \item Sin adición de calor.
    \end{enumerate}

    \item ¿Por qué los motores diésel pueden operar con relaciones de compresión más altas que los motores de gasolina?
    \begin{enumerate}[label=\alph*)]
        \item Porque utilizan un combustible más volátil.
        \item Porque la combustión se inicia por chispa.
        \item \textbf{Porque solo comprimen aire, evitando la autoignición prematura del combustible.}
        \item Porque tienen un sistema de enfriamiento más eficiente.
    \end{enumerate}

    \item ¿Cuál de los siguientes procesos es común tanto en el Ciclo de Otto como en el Ciclo Diesel?
    \begin{enumerate}[label=\alph*)]
        \item Adición de calor a volumen constante.
        \item Adición de calor a presión constante.
        \item \textbf{Expansión adiabática.}
        \item Compresión isotérmica.
    \end{enumerate}

    \item Si un motor de gasolina tiene una relación de compresión muy alta, ¿cómo afecta esto a su eficiencia ideal según el Ciclo de Otto?
    \begin{enumerate}[label=\alph*)]
        \item Disminuye la eficiencia.
        \item \textbf{Aumenta la eficiencia.}
        \item No tiene efecto en la eficiencia.
        \item La eficiencia se vuelve negativa.
    \end{enumerate}

    \item ¿Qué parámetro adicional, además de la relación de compresión, es crucial para la eficiencia del Ciclo Diesel?
    \begin{enumerate}[label=\alph*)]
        \item La temperatura ambiente.
        \item La relación de calores específicos.
        \item \textbf{La relación de corte de inyección.}
        \item El número de cilindros.
    \end{enumerate}

    \item ¿Cuál de los siguientes ciclos es teóricamente más eficiente para una misma relación de compresión?
    \begin{enumerate}[label=\alph*)]
        \item Ciclo de Carnot.
        \item \textbf{Ciclo de Otto.}
        \item Ciclo Diesel.
        \item Todos son igualmente eficientes.
    \end{enumerate}

    \item ¿Qué proceso en un motor de combustión interna es responsable de producir el trabajo útil?
    \begin{enumerate}[label=\alph*)]
        \item Compresión.
        \item Escape.
        \item \textbf{Expansión (carrera de potencia).}
        \item Admisión.
    \end{enumerate}

    \item ¿Cuál es una ventaja clave de los motores diésel en comparación con los de gasolina?
    \begin{enumerate}[label=\alph*)]
        \item Mayor suavidad y silencio de operación.
        \item Mayor potencia específica (por unidad de peso).
        \item \textbf{Mayor eficiencia térmica y economía de combustible.}
        \item Menores emisiones de NOx.
    \end{enumerate}

    \item ¿Qué representa la relación de corte de inyección ($r_c$) en el Ciclo Diesel?
    \begin{enumerate}[label=\alph*)]
        \item La relación entre el volumen máximo y mínimo.
        \item \textbf{La relación entre el volumen al final y al inicio de la combustión a presión constante.}
        \item La relación entre la presión máxima y mínima.
        \item La relación entre las temperaturas de los depósitos.
    \end{enumerate}

\end{enumerate}

\end{document}
