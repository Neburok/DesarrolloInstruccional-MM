\documentclass{article}

% --- Cargar la plantilla de estilo UTP ---
\usepackage{_PLANTILLAS/TEMPLATE_LECTURA_PDF/utp-doc}
\usepackage{amsmath}

\begin{document}

% --- Título de la Guía ---
\practicatitle{Guía de Ejercicios Resueltos: Transferencia de Calor}

% --- Metadatos de la Actividad ---
\textbf{Asignatura:} Termodinámica Automotriz \\
\textbf{Unidad 4:} Procesos Termodinámicos y de Transferencia de Calor

\vspace{5mm}
\hrule
\vspace{5mm}

\section*{Introducción}

Esta guía de ejercicios está diseñada para reforzar su comprensión de los mecanismos de transferencia de calor: conducción, convección y radiación. La resolución de estos problemas le permitirá aplicar las leyes fundamentales y desarrollar sus habilidades de cálculo en contextos relevantes para la ingeniería automotriz. Cada ejercicio incluye una solución detallada paso a paso para facilitar su aprendizaje.

\section*{Ejercicio 1: Conducción a través de una Pared Plana}

Una pared de un cilindro de motor de $0.5 \, cm$ de espesor está hecha de un material con una conductividad térmica de $k = 45 \, W/m \cdot K$. La temperatura de la superficie interior de la pared es de $800^\circ C$ y la de la superficie exterior es de $150^\circ C$. El área de la superficie de la pared es de $0.01 \, m^2$.

Calcule la tasa de transferencia de calor por conducción a través de esta pared.

\subsection*{Solución Detallada}

Para calcular la tasa de transferencia de calor por conducción a través de una pared plana, utilizamos la Ley de Fourier:

$$ Q_{cond} = -k A \frac{dT}{dx} $$

Para una pared plana, el gradiente de temperatura $\frac{dT}{dx}$ se puede aproximar como $\frac{\Delta T}{\Delta x}$, donde $\Delta T = T_{exterior} - T_{interior}$ y $\Delta x$ es el espesor de la pared. Sin embargo, para obtener un valor positivo de $Q_{cond}$ (que representa el flujo de calor desde la temperatura más alta a la más baja), es más práctico usar la diferencia de temperatura positiva $(T_{interior} - T_{exterior})$.

$$ Q_{cond} = k A \frac{T_{interior} - T_{exterior}}{\Delta x} $$

\textbf{Datos:}
\begin{itemize}
    \item Conductividad térmica ($k$) = $45 \, W/m \cdot K$
    \item Espesor ($\Delta x$) = $0.5 \, cm = 0.005 \, m$
    \item Temperatura interior ($T_{interior}$) = $800^\circ C$
    \item Temperatura exterior ($T_{exterior}$) = $150^\circ C$
    \item Área ($A$) = $0.01 \, m^2$
\end{itemize}

\textbf{Cálculo:}

$$ Q_{cond} = (45 \, W/m \cdot K)(0.01 \, m^2) \frac{(800 - 150) \, ^\circ C}{0.005 \, m} $$

$$ Q_{cond} = (0.45 \, W/K) \frac{650 \, ^\circ C}{0.005 \, m} $$

$$ Q_{cond} = 0.45 \times 130,000 \, W = 58,500 \, W = 58.5 \, kW $$

La tasa de transferencia de calor por conducción a través de la pared del cilindro es de $58.5 \, kW$.

\section*{Ejercicio 2: Convección en un Radiador de Automóvil}

Un radiador de automóvil tiene una superficie de $0.8 \, m^2$ y su temperatura superficial promedio es de $90^\circ C$. El aire que fluye a través del radiador está a $25^\circ C$. Si el coeficiente de transferencia de calor por convección es de $120 \, W/m^2 \cdot K$, ¿cuál es la tasa de transferencia de calor por convección del radiador al aire?

\subsection*{Solución Detallada}

Para calcular la tasa de transferencia de calor por convección, utilizamos la Ley de Enfriamiento de Newton:

$$ Q_{conv} = h A (T_s - T_\infty) $$

\textbf{Datos:}
\begin{itemize}
    \item Coeficiente de transferencia de calor por convección ($h$) = $120 \, W/m^2 \cdot K$
    \item Área ($A$) = $0.8 \, m^2$
    \item Temperatura de la superficie ($T_s$) = $90^\circ C$
    \item Temperatura del fluido ($T_\infty$) = $25^\circ C$
\end{itemize}

\textbf{Cálculo:}

$$ Q_{conv} = (120 \, W/m^2 \cdot K)(0.8 \, m^2)(90 - 25) \, ^\circ C $$

$$ Q_{conv} = (96 \, W/K)(65 \, ^\circ C) $$

$$ Q_{conv} = 6240 \, W = 6.24 \, kW $$

La tasa de transferencia de calor por convección del radiador al aire es de $6.24 \, kW$.

\section*{Ejercicio 3: Radiación de un Tubo de Escape}

Un tubo de escape de acero tiene una superficie exterior con una emisividad de $0.75$ y un área de $0.2 \, m^2$. La temperatura de la superficie del tubo es de $400^\circ C$. Los alrededores (ambiente del motor) están a $80^\circ C$.

Calcule la tasa neta de transferencia de calor por radiación desde el tubo de escape a los alrededores.

\subsection*{Solución Detallada}

Para calcular la tasa neta de transferencia de calor por radiación, utilizamos la Ley de Stefan-Boltzmann. Es crucial convertir las temperaturas a Kelvin.

$$ Q_{rad} = \epsilon \sigma A (T_s^4 - T_{alrededores}^4) $$

\textbf{Datos:}
\begin{itemize}
    \item Emisividad ($\epsilon$) = $0.75$
    \item Constante de Stefan-Boltzmann ($\sigma$) = $5.67 \times 10^{-8} \, W/m^2 \cdot K^4$
    \item Área ($A$) = $0.2 \, m^2$
    \item Temperatura de la superficie ($T_s$) = $400^\circ C = 400 + 273.15 = 673.15 \, K$
    \item Temperatura de los alrededores ($T_{alrededores}$) = $80^\circ C = 80 + 273.15 = 353.15 \, K$
\end{itemize}

\textbf{Cálculo:}

$$ Q_{rad} = (0.75)(5.67 \times 10^{-8} \, W/m^2 \cdot K^4)(0.2 \, m^2) ((673.15 \, K)^4 - (353.15 \, K)^4) $$

$$ Q_{rad} = (0.75)(5.67 \times 10^{-8})(0.2) (2.05 \times 10^{11} - 1.56 \times 10^{10}) $$

$$ Q_{rad} = (8.505 \times 10^{-9}) (1.894 \times 10^{11}) $$

$$ Q_{rad} = 1610.8 \, W \approx 1.61 \, kW $$

La tasa neta de transferencia de calor por radiación desde el tubo de escape es de aproximadamente $1.61 \, kW$.

\end{document}
