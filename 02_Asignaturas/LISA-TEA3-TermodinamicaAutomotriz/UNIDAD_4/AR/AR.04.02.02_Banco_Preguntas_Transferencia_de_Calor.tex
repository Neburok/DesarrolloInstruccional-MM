\documentclass{article}

% --- Cargar la plantilla de estilo UTP ---
\usepackage{utp-doc}
\usepackage{tikz}
\usepackage{amsmath}
\usepackage{enumitem}

\begin{document}

% --- Título del Banco de Preguntas ---
\practicatitle{Banco de Preguntas de Opción Múltiple: Transferencia de Calor}

% --- Metadatos de la Actividad ---
\textbf{Asignatura:} Termodinámica Automotriz \\
\textbf{Unidad 4:} Procesos Termodinámicos y de Transferencia de Calor

\vspace{5mm}
\hrule
\vspace{5mm}

\section*{Introducción}

Este banco de preguntas le permitirá evaluar su comprensión de los conceptos clave relacionados con los mecanismos de transferencia de calor: conducción, convección y radiación. Seleccione la opción que considere correcta para cada pregunta.

\section*{Preguntas}

\begin{enumerate}[label=\arabic*.]

    \item ¿Cuál es la principal diferencia entre la termodinámica y la transferencia de calor?
    \begin{enumerate}[label=\alph*)]
        \item La termodinámica estudia la velocidad de transferencia, la transferencia de calor estudia la cantidad.
        \item \textbf{La termodinámica estudia la cantidad de calor, la transferencia de calor estudia la velocidad.}
        \item La termodinámica solo aplica a gases, la transferencia de calor a sólidos.
        \item No hay diferencia, son términos intercambiables.
    \end{enumerate}

    \item ¿Qué mecanismo de transferencia de calor requiere el contacto directo entre partículas?
    \begin{enumerate}[label=\alph*)]
        \item Convección.
        \item Radiación.
        \item \textbf{Conducción.}
        \item Evaporación.
    \end{enumerate}

    \item La Ley de Fourier describe la transferencia de calor por:
    \begin{enumerate}[label=\alph*)]
        \item Convección.
        \item Radiación.
        \item \textbf{Conducción.}
        \item Ebullición.
    \end{enumerate}

    \item En la Ley de Fourier ($Q_{cond} = -k A \frac{dT}{dx}$), ¿qué representa la variable $k$?
    \begin{enumerate}[label=\alph*)]
        \item El coeficiente de convección.
        \item La emisividad del material.
        \item \textbf{La conductividad térmica.}
        \item La constante de Stefan-Boltzmann.
    \end{enumerate}

    \item Un material con una alta conductividad térmica es un buen:
    \begin{enumerate}[label=\alph*)]
        \item Aislante.
        \item \textbf{Conductor de calor.}
        \item Emisor de radiación.
        \item Fluido.
    \end{enumerate}

    \item ¿Qué mecanismo de transferencia de calor implica el movimiento de un fluido (líquido o gas)?
    \begin{enumerate}[label=\alph*)]
        \item Conducción.
        \item Radiación.
        \item \textbf{Convección.}
        \item Absorción.
    \end{enumerate}

    \item La Ley de Enfriamiento de Newton describe la transferencia de calor por:
    \begin{enumerate}[label=\alph*)]
        \item Conducción.
        \item Radiación.
        \item \textbf{Convección.}
        \item Fusión.
    \end{enumerate}

    \item ¿Qué tipo de convección ocurre cuando el movimiento del fluido es causado por diferencias de densidad debido a cambios de temperatura?
    \begin{enumerate}[label=\alph*)]
        \item Convección forzada.
        \item \textbf{Convección natural (o libre).}
        \item Convección inducida.
        \item Convección de fase.
    \end{enumerate}

    \item Un ventilador en un radiador de automóvil es un ejemplo de convección:
    \begin{enumerate}[label=\alph*)]
        \item Natural.
        \item Libre.
        \item \textbf{Forzada.}
        \item Estática.
    \end{enumerate}

    \item ¿Qué mecanismo de transferencia de calor no requiere un medio material para propagarse?
    \begin{enumerate}[label=\alph*)]
        \item Conducción.
        \item Convección.
        \item \textbf{Radiación.}
        \item Todas las anteriores.
    \end{enumerate}

    \item La Ley de Stefan-Boltzmann describe la transferencia de calor por:
    \begin{enumerate}[label=\alph*)]
        \item Conducción.
        \item Convección.
        \item \textbf{Radiación.}
        \item Evaporación.
    \end{enumerate}

    \item En la Ley de Stefan-Boltzmann ($Q_{rad} = \epsilon \sigma A (T_s^4 - T_{alrededores}^4)$), ¿qué representa la variable $\epsilon$?
    \begin{enumerate}[label=\alph*)]
        \item La conductividad térmica.
        \item El coeficiente de convección.
        \item \textbf{La emisividad.}
        \item La constante de Boltzmann.
    \end{enumerate}

    \item ¿Por qué es crucial usar temperaturas absolutas (Kelvin) en la Ley de Stefan-Boltzmann?
    \begin{enumerate}[label=\alph*)]
        \item Porque la radiación solo ocurre a altas temperaturas.
        \item \textbf{Porque la relación es a la cuarta potencia de la temperatura.}
        \item Porque es una convención internacional.
        \item Porque los materiales solo emiten radiación en Kelvin.
    \end{enumerate}

    \item ¿Cuál de los siguientes materiales sería el mejor aislante térmico?
    \begin{enumerate}[label=\alph*)]
        \item Cobre.
        \item Aluminio.
        \item \textbf{Aire (estancado).}
        \item Acero.
    \end{enumerate}

    \item En un motor de combustión interna, ¿cuál es el mecanismo principal de transferencia de calor desde la superficie del bloque del motor al aire circundante?
    \begin{enumerate}[label=\alph*)]
        \item Conducción.
        \item \textbf{Convección.}
        \item Radiación.
        \item Fusión.
    \end{enumerate}

    \item Si la temperatura de una superficie se duplica (en Kelvin), ¿cómo cambia la tasa de transferencia de calor por radiación (asumiendo todo lo demás constante)?
    \begin{enumerate}[label=\alph*)]
        \item Se duplica.
        \item Se cuadruplica.
        \item Se multiplica por ocho.
        \item \textbf{Se multiplica por dieciséis.}
    \end{enumerate}

    \item ¿Qué mecanismo de transferencia de calor es responsable de que sintamos el calor de un tubo de escape caliente sin tocarlo?
    \begin{enumerate}[label=\alph*)]
        \item Conducción.
        \item Convección.
        \item \textbf{Radiación.}
        \item Evaporación.
    \end{enumerate}

    \item ¿Qué factor NO influye directamente en el coeficiente de transferencia de calor por convección ($h$)?
    \begin{enumerate}[label=\alph*)]
        \item La velocidad del fluido.
        \item La geometría de la superficie.
        \item Las propiedades del fluido.
        \item \textbf{La conductividad térmica del sólido.}
    \end{enumerate}

    \item ¿Cuál de las siguientes afirmaciones es correcta sobre la transferencia de calor?
    \begin{enumerate}[label=\alph*)]
        \item El calor siempre fluye de una región fría a una caliente.
        \item La transferencia de calor solo ocurre en sólidos.
        \item \textbf{La energía térmica siempre fluye de una región de mayor temperatura a una de menor temperatura.}
        \item La radiación requiere un medio para propagarse.
    \end{enumerate}

    \item En el contexto de un motor automotriz, ¿por qué es crucial la ciencia de la transferencia de calor?
    \begin{enumerate}[label=\alph*)]
        \item Para determinar la cantidad total de combustible consumido.
        \item \textbf{Para calcular la velocidad a la que se disipa el calor y evitar el sobrecalentamiento.}
        \item Para medir la presión dentro de los cilindros.
        \item Para optimizar la relación de compresión.
    \end{enumerate}

\end{enumerate}

\section*{Clave de Respuestas}
Las respuestas correctas están marcadas en negrita en cada pregunta.

\end{document}
