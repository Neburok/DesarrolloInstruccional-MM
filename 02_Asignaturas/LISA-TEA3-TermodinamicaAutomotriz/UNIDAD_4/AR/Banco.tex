\documentclass{article}

% --- Cargar la plantilla de estilo UTP ---
\usepackage{utp-doc}
\usepackage{tikz}
\usepackage{amsmath}
\usepackage{enumitem}

\begin{document}
	
	% --- Título del Banco de Preguntas ---
	\practicatitle{Banco de Preguntas de Opción Múltiple: Transferencia de Calor}
	
	% --- Metadatos de la Actividad ---
	\textbf{Asignatura:} Termodinámica Automotriz \\
	\textbf{Unidad 4:} Procesos Termodinámicos y de Transferencia de Calor
	
	\vspace{5mm}
	\hrule
	\vspace{5mm}
	
	\section*{Introducción}
	
	Este banco de preguntas le permitirá evaluar su comprensión de los conceptos clave relacionados con los mecanismos de transferencia de calor: conducción, convección y radiación. Seleccione la opción que considere correcta para cada pregunta.
	
	\section*{Preguntas}
	
	\begin{enumerate}[label=\arabic*.]
		
		\item ¿Cuál es la principal diferencia entre la termodinámica y la transferencia de calor?
		\begin{enumerate}[label=\alph*).]
			\item La termodinámica estudia la velocidad de transferencia, la transferencia de calor estudia la cantidad.
			\item \textbf{La termodinámica estudia la cantidad de calor, la transferencia de calor estudia la velocidad.}
			\item La termodinámica solo aplica a gases, la transferencia de calor a sólidos.
			\item No hay diferencia, son términos intercambiables.
		\end{enumerate}
		
		\item ¿Qué mecanismo de transferencia de calor requiere el contacto directo entre partículas?
		\begin{enumerate}[label=\alph*.]
			\item Convección.
			\item Radiación.
			\item \textbf{Conducción.}
			\item Evaporación.
		\end{enumerate}
		
		\item La Ley de Fourier describe la transferencia de calor por:
		\begin{enumerate}[label=\alph*)]
			\item Convección.
			\item Radiación.
			\item \textbf{Conducción.}
			\item Ebullición.
		\end{enumerate}
		
		\item En la Ley de Fourier ($Q_{cond} = -k A \frac{dT}{dx}$), ¿qué representa la variable $k$?
		\begin{enumerate}[label=\alph*)]
			\item El coeficiente de convección.
			\item La emisividad del material.
			\item \textbf{La conductividad térmica.}
			\item La constante de Stefan-Boltzmann.
		\end{enumerate}
		 \item Un material con una alta conductividad térmica es un buen:
		
		\begin{enumerate}[label=\alph*)]
			\item Aislante.
			\item \textbf{Conductor de calor.}
			\item Emisor de radiación.
			\item Fluido.
		\end{enumerate}
			
	    \item ¿Qué mecanismo de transferencia de calor implica el movimiento de un fluido (líquido o gas)?
			\begin{enumerate}[label=\alph*)]
				\item Conducción.
				\item Radiación.
				\item \textbf{Convección.}
				\item Absorción.
			\end{enumerate}
			
	\end{enumerate}
	
\end{document}