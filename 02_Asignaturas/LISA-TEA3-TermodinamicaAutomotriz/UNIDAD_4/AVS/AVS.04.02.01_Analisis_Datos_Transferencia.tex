\documentclass{article}

% --- Cargar la plantilla de estilo UTP ---
\usepackage{utp-doc}
\usepackage{tikz}
\usepackage{amsmath}
\usepackage{enumitem}
\usepackage{tabularx}

\begin{document}

% --- Título de la AVS ---
\practicatitle{Actividad de Verificación de Saberes (AVS): Análisis de Datos de Transferencia de Calor}

% --- Metadatos de la Actividad ---
\textbf{Asignatura:} Termodinámica Automotriz \\
\textbf{Unidad 4:} Procesos Termodinámicos y de Transferencia de Calor \\
\textbf{Tema:} Mecanismos de Transferencia de Calor (Conducción, Convección, Radiación)

\vspace{5mm}
\hrule
\vspace{5mm}

\section*{Objetivo de la Actividad}

Al completar esta actividad, el estudiante será capaz de analizar datos experimentales (o simulados) de sistemas de gestión térmica, aplicar los principios de conducción, convección y radiación para calcular tasas de transferencia de calor, e interpretar los resultados en un formato conciso.

\section*{Instrucciones Generales}

\begin{enumerate}
    \item Esta actividad se realizará de forma individual en el laboratorio o aula de cómputo, con una duración máxima de \textbf{1.5 horas}.
    \item Los datos necesarios para la resolución de los problemas serán proporcionados al inicio de la sesión, ya sea a través de una práctica de laboratorio o de una simulación interactiva de un sistema de transferencia de calor.
    \item Presente todos los cálculos de manera clara y ordenada. Utilice la notación LaTeX para todas las ecuaciones y variables.
    \item Las respuestas deben ser concisas y directas, enfocándose en los resultados numéricos y una breve interpretación.
    \item El entregable será un documento (físico o digital, según se indique) con las soluciones a los problemas planteados.
\end{enumerate}

\section*{Escenario y Problemas a Resolver}

Se ha realizado una prueba en un sistema de enfriamiento de un motor para evaluar la disipación de calor. A continuación, se presentan los datos obtenidos de diferentes componentes. Su tarea es analizar estos datos para cuantificar la transferencia de calor por los distintos mecanismos.

\textbf{Datos Proporcionados (Ejemplo - los datos reales se entregarán en la sesión):}

\begin{itemize}
    \item \textbf{Componente 1: Pared del Bloque del Motor (Conducción)}
    \begin{itemize}
        \item Espesor ($\Delta x$): $0.8 \, cm$
        \item Conductividad térmica ($k$): $50 \, W/m \cdot K$
        \item Temperatura interior ($T_{int}$): $750^\circ C$
        \item Temperatura exterior ($T_{ext}$): $120^\circ C$
        \item Área ($A$): $0.02 \, m^2$
    \end{itemize}

    \item \textbf{Componente 2: Aletas del Radiador (Convección)}
    \begin{itemize}
        \item Temperatura superficial ($T_s$): $85^\circ C$
        \item Temperatura del aire ambiente ($T_\infty$): $20^\circ C$
        \item Área de las aletas ($A$): $1.2 \, m^2$
        \item Coeficiente de transferencia de calor por convección ($h$): $100 \, W/m^2 \cdot K$
    \end{itemize}

    \item \textbf{Componente 3: Tubo de Escape (Radiación)}
    \begin{itemize}
        \item Temperatura superficial ($T_s$): $350^\circ C$
        \item Temperatura de los alrededores ($T_{alrededores}$): $60^\circ C$
        \item Área de la superficie ($A$): $0.15 \, m^2$
        \item Emisividad ($\epsilon$): $0.7$
        \item Constante de Stefan-Boltzmann ($\sigma$): $5.67 \times 10^{-8} \, W/m^2 \cdot K^4$
    \end{itemize}
\end{itemize}

\textbf{Problemas a Resolver:}

\begin{enumerate}
    \item \textbf{Conducción:} Calcule la tasa de transferencia de calor por conducción a través de la pared del bloque del motor. (30\%)
    \item \textbf{Convección:} Calcule la tasa de transferencia de calor por convección desde las aletas del radiador al aire. (30\%)
    \item \textbf{Radiación:} Calcule la tasa de transferencia de calor por radiación desde el tubo de escape a los alrededores. (30\%)
    \item \textbf{Interpretación:} Basado en los resultados, ¿cuál de los tres mecanismos es el más significativo en este escenario y por qué? (10\%)
\end{enumerate}

\vspace{5mm}
\hrule
\vspace{5mm}

\section*{Rúbrica de Evaluación}

\begin{tabularx}{\textwidth}{|X|X|X|X|X|X|X|X|}
\hline
\textbf{Criterio de Evaluación} & \textbf{10 Estratégico (90-100\%)} & \textbf{9 Autónomo (80-89\%)} & \textbf{8 Básico (70-79\%)} & \textbf{7 Receptivo (60-69\%)} & \textbf{6 Preformal (50-59\%)} & \textbf{0 No entrega (0\%)} & \textbf{Puntaje} \\
\hline
\textbf{1. Conducción} & Cálculo preciso y bien presentado. & Cálculo correcto con error menor. & Cálculo con error significativo. & Cálculo incompleto o con error conceptual. & Cálculo incorrecto o ausente. & No entrega. & /30\% \\
\hline
\textbf{2. Convección} & Cálculo preciso y bien presentado. & Cálculo correcto con error menor. & Cálculo con error significativo. & Cálculo incompleto o con error conceptual. & Cálculo incorrecto o ausente. & No entrega. & /30\% \\
\hline
\textbf{3. Radiación} & Cálculo preciso y bien presentado. & Cálculo correcto con error menor. & Cálculo con error significativo. & Cálculo incompleto o con error conceptual. & Cálculo incorrecto o ausente. & No entrega. & /30\% \\
\hline
\textbf{4. Interpretación} & Análisis profundo y justificado. & Análisis adecuado. & Análisis básico. & Análisis superficial. & Análisis incorrecto o ausente. & No entrega. & /10\% \\
\hline
\textbf{Puntaje Total} & & & & & & & /100\% \\
\hline
\end{tabularx}

\end{document}
