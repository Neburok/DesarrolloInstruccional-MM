\documentclass[10pt,aspectratio=169]{beamer}

% Tema y configuración
\usetheme{Madrid}
\usecolortheme{beaver}

% Paquetes
\usepackage[utf8]{inputenc}
\usepackage[spanish]{babel}
\usepackage{amsmath}
\usepackage{amsfonts}
\usepackage{amssymb}
\usepackage{graphicx}
\usepackage{booktabs}
\usepackage{tikz}
\usepackage{xcolor}

% Configuración de colores personalizados
\definecolor{azulUTQ}{RGB}{0,51,102}
\definecolor{grisClaro}{RGB}{245,245,245}
\definecolor{verdeExito}{RGB}{46,125,50}
\definecolor{naranjaAlerta}{RGB}{255,152,0}

% Personalización del tema
\setbeamercolor{structure}{fg=azulUTQ}
\setbeamercolor{frametitle}{bg=azulUTQ,fg=white}
\setbeamercolor{title}{fg=azulUTQ}
\setbeamertemplate{navigation symbols}{}
\setbeamertemplate{itemize items}[triangle]

% Información del documento
\title[Estandarización de Materiales Educativos]{%
	Proyecto de Estandarización\\
	de Materiales Educativos}
\subtitle{Sistematización del Desarrollo de\\Actividades de Desarrollo (ADs)}
\author{Equipo de Desarrollo Pedagógico}
\institute{Universidad Tecnológica de Querétaro}
\date{\today}

% Logo en el pie de página
%\logo{\includegraphics[height=0.8cm]{placeholder_logo_utq.png}}

\begin{document}
	
%%%%%%%%%%%%%%%%%%%%%%%%%%%%%%%%%%%%%%%%%%%%%%%%%%%%%%%%%%%%%%%%%%%%%%%%%
% PORTADA
%%%%%%%%%%%%%%%%%%%%%%%%%%%%%%%%%%%%%%%%%%%%%%%%%%%%%%%%%%%%%%%%%%%%%%%%%
	
\begin{frame}
	\titlepage
	\begin{center}
		\small
		\textit{Una propuesta metodológica para garantizar calidad,\\
				coherencia y escalabilidad en el desarrollo de recursos didácticos}
	\end{center}
\end{frame}
	
	%%%%%%%%%%%%%%%%%%%%%%%%%%%%%%%%%%%%%%%%%%%%%%%%%%%%%%%%%%%%%%%%%%%%%%%%%
	% ÍNDICE
	%%%%%%%%%%%%%%%%%%%%%%%%%%%%%%%%%%%%%%%%%%%%%%%%%%%%%%%%%%%%%%%%%%%%%%%%%
	
	\begin{frame}{Agenda}
		\tableofcontents
	\end{frame}
	
	%%%%%%%%%%%%%%%%%%%%%%%%%%%%%%%%%%%%%%%%%%%%%%%%%%%%%%%%%%%%%%%%%%%%%%%%%
	% SECCIÓN 1: SITUACIÓN ACTUAL
	%%%%%%%%%%%%%%%%%%%%%%%%%%%%%%%%%%%%%%%%%%%%%%%%%%%%%%%%%%%%%%%%%%%%%%%%%
	
	\section{Situación Actual}
	
	\begin{frame}{Desafíos Identificados}
		\begin{columns}
			\begin{column}{0.5\textwidth}
				\begin{block}{Falta de Estandarización}
					\begin{itemize}
						\item Inconsistencias en formato
						\item Variabilidad en estructura
						\item Calidad heterogénea
					\end{itemize}
				\end{block}
				
				\begin{block}{Proceso No Sistemático}
					\begin{itemize}
						\item Desarrollo ad-hoc
						\item Sin metodología clara
						\item Ausencia de criterios
					\end{itemize}
				\end{block}
			\end{column}
			
			\begin{column}{0.5\textwidth}
				\begin{block}{Ineficiencia Operativa}
					\begin{itemize}
						\item Duplicación de esfuerzos
						\item Reinvención de recursos
						\item Pérdida de tiempo
					\end{itemize}
				\end{block}
				
				\begin{block}{Falta de Evaluación}
					\begin{itemize}
						\item Sin criterios objetivos
						\item Calidad no medible
						\item Mejora no sistemática
					\end{itemize}
				\end{block}
			\end{column}
		\end{columns}
	\end{frame}
	
	%%%%%%%%%%%%%%%%%%%%%%%%%%%%%%%%%%%%%%%%%%%%%%%%%%%%%%%%%%%%%%%%%%%%%%%%%
	% SECCIÓN 2: PROPUESTA
	%%%%%%%%%%%%%%%%%%%%%%%%%%%%%%%%%%%%%%%%%%%%%%%%%%%%%%%%%%%%%%%%%%%%%%%%%
	
	\section{Nuestra Propuesta}
	
	\begin{frame}{Solución Integral}
		\begin{center}
			\Large
			\textcolor{azulUTQ}{\textbf{Guía Metodológica Integral}}\\
			\vspace{0.5cm}
			\normalsize
			para el desarrollo estandarizado de\\
			Actividades de Desarrollo (ADs)
		\end{center}
		
		\vspace{1cm}
		
		\begin{columns}
			\begin{column}{0.5\textwidth}
				\begin{itemize}
					\item \textbf{Metodología estructurada} en 4 fases
					\item \textbf{Control de calidad} por pares
				\end{itemize}
			\end{column}
			\begin{column}{0.5\textwidth}
				\begin{itemize}
					\item \textbf{Herramientas estandarizadas}
					\item \textbf{Proceso escalable} y replicable
				\end{itemize}
			\end{column}
		\end{columns}
	\end{frame}
	
	\begin{frame}{Metodología de Cuatro Fases}
		\begin{center}
			\begin{tikzpicture}[scale=0.8]
				% Fase 1
				\node[rectangle, draw=azulUTQ, fill=azulUTQ!20, minimum width=3cm, minimum height=1.5cm] at (0,0) {
					\begin{minipage}{2.8cm}
						\centering
						\textbf{Fase 1}\\
						\small Análisis y\\Diseño Pedagógico
					\end{minipage}
				};
				
				% Fase 2
				\node[rectangle, draw=azulUTQ, fill=azulUTQ!20, minimum width=3cm, minimum height=1.5cm] at (4,0) {
					\begin{minipage}{2.8cm}
						\centering
						\textbf{Fase 2}\\
						\small Creación y\\Redacción
					\end{minipage}
				};
				
				% Fase 3
				\node[rectangle, draw=azulUTQ, fill=azulUTQ!20, minimum width=3cm, minimum height=1.5cm] at (8,0) {
					\begin{minipage}{2.8cm}
						\centering
						\textbf{Fase 3}\\
						\small Estandarización\\y Organización
					\end{minipage}
				};
				
				% Fase 4
				\node[rectangle, draw=azulUTQ, fill=azulUTQ!20, minimum width=3cm, minimum height=1.5cm] at (12,0) {
					\begin{minipage}{2.8cm}
						\centering
						\textbf{Fase 4}\\
						\small Validación\\y Entrega
					\end{minipage}
				};
				
				% Flechas
				\draw[->, thick, azulUTQ] (1.5,0) -- (2.5,0);
				\draw[->, thick, azulUTQ] (5.5,0) -- (6.5,0);
				\draw[->, thick, azulUTQ] (9.5,0) -- (10.5,0);
			\end{tikzpicture}
		\end{center}
		
		\vspace{0.5cm}
		
		\begin{itemize}
			\item \textbf{Fase 1:} Consulta del programa y selección metodológica
			\item \textbf{Fase 2:} Desarrollo de contenido con formato LaTeX
			\item \textbf{Fase 3:} Aplicación de nomenclaturas y estructura
			\item \textbf{Fase 4:} Revisión por pares y generación de PDF final
		\end{itemize}
	\end{frame}
	
	%%%%%%%%%%%%%%%%%%%%%%%%%%%%%%%%%%%%%%%%%%%%%%%%%%%%%%%%%%%%%%%%%%%%%%%%%
	% SECCIÓN 3: COMPONENTES
	%%%%%%%%%%%%%%%%%%%%%%%%%%%%%%%%%%%%%%%%%%%%%%%%%%%%%%%%%%%%%%%%%%%%%%%%%
	
	\section{Ecosistema de Materiales}
	
	\begin{frame}{Paquete Integral por Unidad}
		\begin{center}
			\textbf{Cada unidad académica contará con cuatro componentes:}
		\end{center}
		
		\vspace{0.5cm}
		
		\begin{columns}
			\begin{column}{0.5\textwidth}
				\begin{block}{AD - Actividades de Desarrollo}
					Material principal para aprendizaje autónomo
				\end{block}
				
				\begin{block}{AR - Actividades de Reforzamiento}
					Ejercicios complementarios para práctica
				\end{block}
			\end{column}
			
			\begin{column}{0.5\textwidth}
				\begin{block}{AVS - Verificación de Saberes}
					Evaluación sumativa presencial
				\end{block}
				
				\begin{block}{ER - Evaluación de Recuperación}
					Evaluación alternativa de recuperación
				\end{block}
			\end{column}
		\end{columns}
		
		\vspace{0.5cm}
		
		\begin{center}
			\textcolor{verdeExito}{\textbf{Cobertura completa del proceso educativo}}
		\end{center}
	\end{frame}
	
	%%%%%%%%%%%%%%%%%%%%%%%%%%%%%%%%%%%%%%%%%%%%%%%%%%%%%%%%%%%%%%%%%%%%%%%%%
	% SECCIÓN 4: BENEFICIOS
	%%%%%%%%%%%%%%%%%%%%%%%%%%%%%%%%%%%%%%%%%%%%%%%%%%%%%%%%%%%%%%%%%%%%%%%%%
	
	\section{Beneficios Institucionales}
	
	\begin{frame}{Ventajas Estratégicas}
		\begin{columns}
			\begin{column}{0.5\textwidth}
				\begin{alertblock}{Calidad Garantizada}
					\begin{itemize}
						\item Precisión técnica validada
						\item Efectividad pedagógica
						\item Revisión por expertos
					\end{itemize}
				\end{alertblock}
				
				\begin{alertblock}{Eficiencia Operativa}
					\begin{itemize}
						\item Reducción de tiempo
						\item Optimización de recursos
						\item Eliminación de duplicidades
					\end{itemize}
				\end{alertblock}
			\end{column}
			
			\begin{column}{0.5\textwidth}
				\begin{alertblock}{Imagen Institucional}
					\begin{itemize}
						\item Profesionalización
						\item Uniformidad de materiales
						\item Fortalecimiento del prestigio
					\end{itemize}
				\end{alertblock}
				
				\begin{alertblock}{Escalabilidad}
					\begin{itemize}
						\item Metodología replicable
						\item Crecimiento sostenido
						\item Adaptabilidad futura
					\end{itemize}
				\end{alertblock}
			\end{column}
		\end{columns}
	\end{frame}
	
	%%%%%%%%%%%%%%%%%%%%%%%%%%%%%%%%%%%%%%%%%%%%%%%%%%%%%%%%%%%%%%%%%%%%%%%%%
	% SECCIÓN 5: INNOVACIÓN
	%%%%%%%%%%%%%%%%%%%%%%%%%%%%%%%%%%%%%%%%%%%%%%%%%%%%%%%%%%%%%%%%%%%%%%%%%
	
	\section{Innovación Metodológica}
	
	\begin{frame}{Diferenciadores Únicos}
		\begin{itemize}
			\item \textbf{Doble Validación:}
			\begin{itemize}
				\item \textit{Pares Colaboradores:} Coherencia metodológica
				\item \textit{Pares Expertos:} Precisión técnica industrial
			\end{itemize}
			
			\vspace{0.3cm}
			
			\item \textbf{Precisión Técnica:}
			\begin{itemize}
				\item Formato LaTeX obligatorio para expresiones matemáticas
				\item Nomenclatura estandarizada de archivos
			\end{itemize}
			
			\vspace{0.3cm}
			
			\item \textbf{Contextualización Nacional:}
			\begin{itemize}
				\item Ejemplos específicos de la industria mexicana
				\item Aplicaciones relevantes al contexto local
			\end{itemize}
			
			\vspace{0.3cm}
			
			\item \textbf{Mejora Continua:}
			\begin{itemize}
				\item Ciclo de retroalimentación estructurado
				\item Evolución constante del proceso
			\end{itemize}
		\end{itemize}
	\end{frame}
	
	%%%%%%%%%%%%%%%%%%%%%%%%%%%%%%%%%%%%%%%%%%%%%%%%%%%%%%%%%%%%%%%%%%%%%%%%%
	% SECCIÓN 6: IMPLEMENTACIÓN
	%%%%%%%%%%%%%%%%%%%%%%%%%%%%%%%%%%%%%%%%%%%%%%%%%%%%%%%%%%%%%%%%%%%%%%%%%
	
	\section{Implementación}
	
	\begin{frame}{Cronograma de Implementación}
		\begin{center}
			\begin{tabular}{@{}p{2cm}p{3cm}p{6cm}@{}}
				\toprule
				\textbf{Período} & \textbf{Fase} & \textbf{Actividades} \\
				\midrule
				Mes 1-2 & Capacitación & Formación del equipo docente en la metodología \\
				\midrule
				Mes 3-4 & Piloto & Desarrollo de ADs para 2-3 asignaturas clave \\
				\midrule
				Mes 5-6 & Evaluación & Análisis de resultados y ajustes metodológicos \\
				\midrule
				Mes 7-12 & Escalamiento & Implementación en todas las asignaturas \\
				\bottomrule
			\end{tabular}
		\end{center}
		
		\vspace{0.5cm}
		
		\begin{center}
			\textcolor{verdeExito}{\textbf{Implementación gradual y controlada}}
			\end{center>
			\end{frame}
			
			\begin{frame}{Recursos Requeridos}
				\begin{columns}
					\begin{column}{0.5\textwidth}
						\begin{block}{Recursos Humanos}
							\begin{itemize}
								\item Coordinador del proyecto
								\item Equipo de revisores pares
								\item Docentes participantes
								\item Expertos industriales
							\end{itemize}
						\end{block}
					\end{column}
					
					\begin{column}{0.5\textwidth}
						\begin{block}{Recursos Técnicos}
							\begin{itemize}
								\item Herramientas LaTeX/Markdown
								\item Sistema de gestión documental
								\item Plataforma de colaboración
								\item Templates y plantillas
							\end{itemize}
						\end{block}
					\end{column}
				\end{columns}
				
				\vspace{0.5cm}
				
				\begin{center}
					\textcolor{azulUTQ}{\textbf{Inversión inicial mínima para máximo impacto}}
					\end{center>
						\end{frame>
							
							%%%%%%%%%%%%%%%%%%%%%%%%%%%%%%%%%%%%%%%%%%%%%%%%%%%%%%%%%%%%%%%%%%%%%%%%%
							% SECCIÓN 7: CONCLUSIÓN
							%%%%%%%%%%%%%%%%%%%%%%%%%%%%%%%%%%%%%%%%%%%%%%%%%%%%%%%%%%%%%%%%%%%%%%%%%
							
							\section{Llamada a la Acción}
							
							\begin{frame}{Es Momento de Liderar el Cambio}
								\begin{center}
									\Large
									\textcolor{azulUTQ}{\textbf{Esta propuesta posiciona a nuestra institución}}\\
									\textcolor{azulUTQ}{\textbf{como pionera en sistematización educativa}}
									
									\vspace{1cm}
									
									\normalsize
									\begin{block}{Pregunta Clave}
										\centering
										¿Están listos para transformar\\
										nuestro proceso educativo?
									\end{block}
									
									\vspace{0.5cm>
										
										\textit{``La excelencia educativa no es un accidente.\\
											Es el resultado de un proceso intencional,\\
											sistemático y bien ejecutado.''}
										\end{center>
											\end{frame>
												
												\begin{frame}{Próximos Pasos}
													\begin{enumerate}
														\item \textbf{Aprobación del proyecto} por parte de la dirección
														
														\vspace{0.3cm}
														
														\item \textbf{Asignación de recursos} humanos y técnicos
														
														\vspace{0.3cm}
														
														\item \textbf{Conformación del equipo} de trabajo
														
														\vspace{0.3cm}
														
														\item \textbf{Inicio de la fase piloto} con asignaturas seleccionadas
														
														\vspace{0.3cm}
														
														\item \textbf{Evaluación y ajustes} basados en resultados iniciales
														\end{enumerate>
															
															\vspace{1cm}
															
															\begin{center}
																\textcolor{verdeExito}{\Large \textbf{¡Estamos listos para comenzar!}}
																\end{center>
																	\end{frame>
																		
																		%%%%%%%%%%%%%%%%%%%%%%%%%%%%%%%%%%%%%%%%%%%%%%%%%%%%%%%%%%%%%%%%%%%%%%%%%
																		% SLIDE FINAL
																		%%%%%%%%%%%%%%%%%%%%%%%%%%%%%%%%%%%%%%%%%%%%%%%%%%%%%%%%%%%%%%%%%%%%%%%%%
																		
																		\begin{frame}{}
																			\begin{center}
																				\Huge
																				\textcolor{azulUTQ}{\textbf{¿Preguntas?}}
																				
																				\vspace{1cm}
																				
																				\Large
																				Gracias por su atención
																				
																				\vspace{0.5cm}
																				
																				\normalsize
																				Equipo de Desarrollo Pedagógico\\
																				Universidad Tecnológica de Querétaro
																				\end{center>
																				\end{frame}
																				
																				\end{document>